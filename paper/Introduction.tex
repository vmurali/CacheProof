h\section{Introduction}
\label{sec:Introduction}

Shared memory is a common abstraction for distributed systems as it
provides an easily understood model of interaction. To achieve this
abstraction efficiently, many hardware systems implement a cache
coherence implementation. 

Since the correctness of this abstraction is paramount in the correct
functioning of the system, significant effort is placed in making sure
that the cache coherence engine is correct. In practice, this is done
in two steps. First, a cache coherence protocol is constructed and
formally verified. This protocol generally directly represents many of
the key concurrent aspects of an implementation, \eg\ reads and
writes to the system are broken into local atomic microoperations, but
ignores implementation details \eg\ buffer sizing and memory access
times. Second, once the protocol has been verified it is handed to
hardware experts who fill in the missing implementation details and
realize it at the Register Transfer Level (RTL) of gates, wires, and
registers. This effort is not a one time translation, as performance
properties are not easily estimated; a significant amount of
performance testing is requires to select a good implementation. 

From a methodological perspective, this two step process represent a
significant methodological failure. We spent a significant amount of
time verifying the protocol and then invalidate any correctness
guarantees to the implementation by performing not only a non-trivial
translation from the protocol representation to hardware, but also
the refinement to add missing implementation details.

Previously, it was shown that, because verified protocols are often
described in as via Term Rewrite Systems (TRS), one could directly
transliterate a verified protocol into Bluespec, a hardware
description language on Guarded Atomic Actions, a restriction of
TRS. This reduces the correctness of the implementation to the
correctness of the protocol, effectively removing any implementation
bugs. This requires the protocol include many implementation-level
details which had previously been elided for simplicity from the
protocol so that it is realizable; practically, beyond some additional
descriptional noise in the proof this is not a show stopper.

The issue with this approach is that design of a hardware system
requires many design tweaks to reach optimal performance as one cannot
practically know the correct low-level hardware results without
evaluating the low-level representation. In this protocol
transliteration scheme, each of these changes can potentially breaks
correctness and a full reverification of correctness is needed in
principle. As hardware designers are unlikely to have the expertise or
skill this effectively nullifies the verification guarantee in design.

To fully realize this approach of design we need to provide not just a
single proof of correctness, but a proof framework covering any
variation of a design that might be of interest for performance
reasons. This would enable implementation level changes without
compromising on the ability to verify correctness of the design.

In practice, protocol variants rely on a small set of general
``building block'' properties, \eg\ messages always get from source to
destination or that the design's reaction to a protocol message
results in the appropriate messages being sent, about parts of the
design to be true for the full system correctness. The details of
these properties hold vary depend on the precise design variant. For
instance, to show a message is handled in a finite amount of time it
is sufficient to check that there is a maximum time bound between
start and completion; this may rely on the number of messages in
flight which is in turn affected by buffering size.

In this paper, we describe an approach to provide this proof framework
for any particular cache coherence protocol via the example of a
MOESI-style protocol. We show of the modular structure of the proof
and design can be shared and how modification of microarchtectural
aspects of design directly translation to reverification of our key
building block properties which are effectively trivial for the
microarchitectural variations required. 

