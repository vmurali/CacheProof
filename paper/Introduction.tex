\section{Introduction}
\label{sec:Introduction}

Shared memory is a common abstraction for distributed systems, as it provides
an easily understood model of interaction. To achieve this abstraction
efficiently, many hardware systems implement \emph{cache coherence
protocols}.

Since the correctness of this abstraction is paramount in the correct
functioning of the system, significant effort is expended in making sure that
the cache coherence engine is correct. In practice, verification proceeds in
two steps. First, a cache coherence protocol is constructed and verified. This
protocol generally directly represents many of the key concurrent aspects of an
implementation (\eg{} reads and writes to the system are broken into local
atomic micro-operations) but ignores implementation details (\eg{} buffer
sizing and memory access times). Second, once the protocol has been verified,
it is handed to hardware experts who fill in the missing implementation details
and realize it at the Register Transfer Level (RTL) of gates, wires, and
registers, via further breaking down of local micro-operations to make them
realizable in hardware. This effort is not a one-time translation, as
performance properties are not easily estimated; a significant amount of
performance testing is requires to select a good implementation. 

\adamc{I don't understand what this paragraph is saying.}
From a methodological perspective, this two-step process leaves much to be
desired. The significant effort of verification is effectively lost by the
concurrent non-trivial tasks of translation from the protocol representation to
a hardware description and the necessary refinement to add missing
implementation details to the protocol. 

Verified protocols are often described via Term Rewrite
Systems (TRS) \adamc{citation for TRSes?}, which apply equational logic to describe systems at an
interesting point on the spectrum between declarative and operational methods.
It was shown by Dave
\etal{}~\cite{DNA:CoherenceImplementation} that such a verified
protocol could be transliterated into Bluespec~\cite{Bluespec:TFRG}, a
hardware description language based on Guarded Atomic Actions, a
restriction of TRS~\cite{Hoe:TCAD}. In this way one reduces the correctness of
the implementation to the correctness of the protocol, with no need
for separate reasoning about an implementation. While the current Bluespec compiler
does not support some implementation elements that are common in practice
(\eg{} fairness guarantees, atomic actions running over multiple clock cycles),
recent work has provided key insights toward lifting the
restrictions~\cite{DNA:CoherenceImplementation, Karczmarek}.  We
believe that Bluespec and its future evolutions represent a
very promising platform for productive implementation and
verification of hardware components, including cache-coherent
shared memories.

Even though the protocol transliteration approach cuts out one
layer of verification effort, it may not be practical in practice.
Design of a hardware system requires many tweaks to reach
optimal performance, since low-level performance details often
can only be found empirically.  Each performance tweak potentially
breaks correctness, or at least breaks the structure of the old
\emph{proof} of correctness, so that a full \emph{reverification}
of correctness is needed in principle. Hardware designers are
unlikely to have the expertise or skill to do each such
verification.  Our motivating question in this paper is, \emph{can
we modularize formal proofs of cache coherence protocols so that
verification experts do the hard work once and for all, and then
hardware designers can instantiate the results to different designs
with much less effort?}

In practice, protocol variants rely on a small set of general ``local
building block'' properties, like guaranteed delivery of messages sent
between processors, or guaranteed state changes in processors upon
receiving certain messages.  These properties are
easily verifiable, since they correspond to a small partition of the
system, for instance a single cache.  The details of how these
properties hold vary depending on the precise design variant. For
instance, the reason why some property holds, like the point-to-point
FIFO ordering for messages transferred between two nodes, depends on
the type of network; the proof will be very different for a mesh
network compared to a point-to-point network.

In this paper, we describe an approach to provide this proof framework
for any particular cache coherence protocol via the example of an
MSI-style protocol.  We also show how it can be extended to other protocols
like MESI and MOESI.  We present a modular proof architecture,
where modification of microarchitectural
aspects of design only necessitates reverification of relatively shallow
local building block properties. Our framework and several instantiations
for it are mechanized in the Coq proof assistant system, providing the
highest levels of formal assurance upon a small trusted code base.
We implemented and verified one particular MSI protocol via an
operational semantics in Coq.

\noindent\textbf{Paper Outline}. In Section~\ref{Sec:Background} we
discuss our notion of correctness for our cache coherence protocol
framework. We describe our MSI protocol example and its implementation
in Section~\ref{Sec:System}. In Section~\ref{Sec:ProofStructure} we discuss
the structuring of our proof framework and verification of the base
protocol. We show implementation refinements in
Section~\ref{Sec:Refinements} and conclude with a general discussion
of the applicability of our technique.
