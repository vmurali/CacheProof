\section{System}
\label{sec:System}

We have chosen as our concrete example is a hierarchical
MSI protocol~\cite{MSI}. As we will
discuss later, this can naturally extended to MESI and MOESI based protocols.
Conceptually, our cache coherence system is
organized as in a tree-like structure with the root node as the main
memory, the leaves as the last local cache for individual processors
interfaces and the remaining nodes as intermediate cache structures
(see Figure~\ref{hier}). Whether a cache is a leaf or not is denoted
by the relation \leaf$(c)$.  The relation $\parent(c, p)$
denotes that $p$ is the parent of cache $c$.

Input and output happen at the leaves in the form of memory requests
issued by and responses issued to the associated processor. These will be
represented by two mappings one from $\mathbb{N}$ to requests and the other
from $\mathbb{N}$ to responses. For each leaf $c$, $c.\pos$ represents the
position of the next request to process; it is set to $0$ initially.

Each node maintains the a permission state for
each address, denoting what operations may be performed on the
data. An address may be in the modified state (\Mo) representing that
it may be both read and modified, or in the shared state (\Sh) where
it may only be read, or in the Invalid state (\In), meaning it can be
neither modified or read. This forms a natural total order of
permissions ($\In < \Sh < \Mo$). For leaf caches, the meaning of these
permission states is straightforward, a load request can be processed only if the
leaf is at least in the \Sh{} state, and a store request can be processed only
if it is in the \Mo{} state. For the non-leaf cache, these states denote the
highest state present in the subtree rooted at this cache, denoting that some
cache in the subtree has the specified permissions (we are modeling inclusive
caches). Of course, all these invariants have to be enforced by the protocol.

Each cache in addition to the data associated with each address $a$
(denoted by $c.\data[a]$) also has a \emph{coherence state}
associated with the address (denoted by $c.\state[a]$) holding the
permissions on that address (\Mo, \Sh, or \In). If a cache does not have a
particular location, the it is equivalent to its permission being \In.
Initially, all addresses in each cache is in the \In state, save for those in
the root cache which are in the \Mo state.

Non-leaf nodes also keep a copy of the coherence state of each of its
children in a \emph{directory}, which contains a the state of each of
its children caches for a particular address. For a non-leaf cache
$p$, the state in its directory for child cache $c$ is denoted by
$p.\dstate[c][a]$, for address $a$. This local version may be stale,
in that the child may have changed it state but the parent has not yet
updated itself. However, the parent state will always be a
conservative over-approximation of the child state.

In order for a cache to change to a higher state for a given address,
\ie{} \emph{upgrade}, it has to \emph{request} its parent.  For a
cache to change to a lower state, \ie{} \emph{downgrade}, it has to
notify its parent cache indicating the downgrade -- it uses
\emph{response} messages for this purpose. Conversely, for a parent
cache to force a child cache to downgrade the child's state for an
address (in order to give permissions to a different child cache), or
equivalently, to downgrade the parent's directory state for the child,
it has to \emph{request} that child cache to downgrade, and a parent
can notify a child about upgrading the child's state, or equivalently,
about upgrading the parent's directory state for the child, by sending
a \emph{response} message.

Each cache is connected to its parent cache using 3 logical FIFO
connections: a) \cpReq{} channel to transmit requests from the child
to the parent, b) \cpResp{} channel to transmit responses from the
child to the parent, and c) \pc{} channel to transmit both request and
response messages from parent to child. 

Since we have dedicated channel between each parent-child pair, we
will index the three channels (\cpReq{}, \cpResp{} and \pc{}) by the
name of the child that uses that channel (using the notation
$\cpReq[c]$, \etc for the \cpReq{} channel associated with cache $c$).
Figure \ref{format} shows the message format transmitted in the 3
channels between caches. In each of the channels $ch$, a message $m$
can be enqueued ($\enq(ch, m)$) or dequeued ($\deq(ch)$), and the
channels can be examined for the first element ($\first(ch)$) or for
availability of any message in the channel ($\avail(ch)$).  Each of
these channels start out empty \ie $\forall ch, \neg \avail(ch)$,
initially.

\begin{figure}
\begin{subfigure}{6.8cm}
\begin{tabular}{|lp{5.8cm}|}
\hline
\multicolumn{2}{|c|}{\Reqcp}\\
\hline
\from: & Current state of child\\
\myto: & Desired (higher) state of child\\
\addr: & Address associated with the request\\
\hline
\hline
\multicolumn{2}{|c|}{\Respcp}\\
\hline
\from: & Current state of child\\
\myto: & Downgraded state of child\\
\addr: & Address associated with the request\\
\data: & Data associated with the address, if necessary\\
\hline
\end{tabular}
\end{subfigure}
\begin{subfigure}{5.4cm}
\begin{tabular}{|lp{4.4cm}|}
\hline
\multicolumn{2}{|c|}{\Mpc}\\
\hline
\typ: & \Req{} (for requests) and \Resp{} (for responses)\\
\myto: & Desired (lower) directory state of a child for \Req{} and upgraded
directory state of a child for \Resp{}\\
\addr: & Address associated with the request\\
\data: & Data associated with the address if necessary (for \Resp{} only)\\
\hline
\end{tabular}
\end{subfigure}
\caption{Data types for messages between caches}
\label{format}
\end{figure}

In addition to the actual coherence states, each cache also maintains
a temporary \emph{wait state} for each address to deal with
transitioning between permission levels. The wait state consists of
two parts: a) a boolean representing whether the cache has sent a
request to its parent cache earlier and is waiting for a response
(denoted by $c.\wait[a]$ for address $a$) and b) one of the three
values \Mo, \Sh{} or \In representing the state upgrade that the cache
is waiting to make (denoted by $c.\waitS[a]$ for address
$a$). Similarly, each non-leaf cache has the corresponding wait states
for its directory.  $p.\dwait[c][a]$ denotes that cache $p$ is waiting
for a response from its child $c$ for address $a$, and
$p.\dwaitS[c][a]$ denotes the state to which $p$ is waiting
for $c$ to downgrade. Both \wait{} and \dwait{} are \False{} initially
for each of the caches, and each of the addresses, because no cache is
waiting for any response from either its child or its parent
initially.

The behavior of the system is described via the atomic state
transitions. These transitions have two parts: the guard dictating
when the transition can fire, and the action defining how the state of
the system changes when the transition happens. In our generated
implementation, these transitions or \emph{guarded atomic actions}
happen concurrently in a single hardware clock, but they will
logically happen happen one at a time. Figure~\ref{trans} gives the
atomic transitions for an MSI-based cache coherent memory subsystem. The actions
in the transitions represent only the specific system states that
change. We use $x \Leftarrow y$ to indicate setting of a specific
state $x$ of the overall system to value $y$ because of a transition.

Figure~\ref{childside} shows the transitions that result in an upgrade of the
state of a cache. It first involves a cache sending an upgrade request to the
its parent when the cache is not already waiting for a response for that
address from the parent (ChildSendReq). This request is received by the parent
(ParentRecvReq), and if the rest of the parent's children caches are
\emph{compatible}, and the parents state also has the permission being
requested, it sends an upgrade response to the cache, while changing the
appropriate directory state. Compatibility of an upgrade request with respect
to a cache's siblings can be defined as follows:
\begin{multline*}
\compat = \lambda (p \in Cache) (c \in Cache) (a \in \textit{Addr}) (s \in
\{\Mo, \Sh, \In\}) \Rightarrow\\ \forall i, i \ne c @->
\mylet s_i := p.\dstate[i][a] \myin\\
 (s = \Mo @-> s_i = \In) \wedge (s = \Sh @-> s_i = \Sh) \wedge (s = \In @-> s_i = \Mo)
\end{multline*}
Finally, the child cache receives the response from the parent (ChildRecvResp)
and sets its state while resetting its wait state if the desired upgrade is
indicated by the response.

Figure~\ref{parentside} shows the transitions that result in a downgrade of the
state of a cache because of a request by the cache's parent when the parent is
not already waiting for a response from that cache (ParentSendReq).  This
request is received by the child (ChildRecvReq), and if the state of each of
the child's own children is not greater than the requested downgrade, the cache
downgrades its state and sends a notification response to the parent. Finally,
the parent receives the response from the cache (ParentRecvResp) and sets the
appropriate directory state while resetting the appropriate directory wait state
if the desired downgrade is indicated by the response.

Figure~\ref{childextra} shows transitions that happen in a non-root cache
because of the presence of \emph{voluntary} responses. Whenever a cache runs
out of space to store the data corresponding to an address, it evicts the data
corresponding to one or more addresses to make room for the new address. This
eviction causes the state of the cache for that address to go to $\In$, which
has to be informed to its parent. We use the same \cpResp{} FIFO (and the same
response format) to indicate this. Such a voluntary response for an address can
take place only if the cache is not waiting for a response from its parent for
that address (ChildVolResp).

Because of the presence of voluntary response, a child cache could have already
downgraded to a state required by a downgrade request from its parent. In such
a case, the cache simply drops the downgrade request (ChildDropReq).

Figure~\ref{procside} shows the transitions that actually serve a processors
request in the leaf caches. LoadReq and StoreReq show the processing of a load
and store request, respectively, from a processor. In the case of a store
request, the data in the cache for the requested address is updated with that
supplied in the request. We do not explicitly show the response generated by
the cache in case of a load request in our memory subsystem model; in a real
system, the data in the cache for the requested address is sent back to the
processor. Our definition of store atomicity requires responses for both load
and store requests, and the response field contains the
$(\textit{Processor}*\mathbb{N})$ pair identifying the request corresponding to
the response, the data from the cache in case of a load response, and a value
for the field \tme. A \Response{} is defined whenever LoadReq or StoreReq
transition takes place. Let $c$ be a leaf cache, and $t$ be the position of a
LoadReq$(c)$ or StoreReq$(c)$ transition in the ordered list of transitions of
the full system starting from the initial state. The fields of the response are
as follows:
\begin{multline*}
\small\langle (c, c.\pos), t, \mylet q := \reqFn(c, c.\pos) \myin\\
\small\myif q.\desc = \Ld \mythen c.\data[q.\addr] \myelse \_ \rangle
\end{multline*}


\begin{figure}
\small
\centering
\begin{subfigure}{\textwidth}
\centering
\begin{tabular}{|ll|}
\hline
\multicolumn{2}{|l|}{\textbf{ChildSendReq}$(c, x, a)$: Child $c$ sending request to upgrade to $x$ for address $a$}\\
\hline
Guard: & $c.\state[a] < x \wedge c.\wait[a] = \False$\\
\hline
Action: & $c.\waitS[a] \Leftarrow x$; $c.wait[a] \Leftarrow \True$; $\enq(\cpReq[c], \langle c.\state[a], x, a \rangle)$;\\
\hline
\hline
\multicolumn{2}{|l|}{\textbf{ParentRecvReq}$(p, c)$: Parent $p$ receiving a request from child $c$}\\
\hline
Guard: & 
$\parent(c,p) \wedge \avail(\cpReq[c]) \wedge \mylet q := \first(\cpReq[c]) \myin$\\
& $p.\dwait[c][q.\addr] = \False \wedge p.\dstate[c][q.\addr] \le q.\from \wedge$\\
& $\compat(p, c, q.\addr, q.\myto) \wedge q.\myto \le p.\state[q.\addr]$\\
\hline
Action: & $\enq(\pc[c], \langle \Resp, q.\myto, q.\addr, $\\
& $\;\;\;%
\myif p.\dstate[c][q.\addr] = \In \mythen p.\data[q.\addr] \myelse \_\rangle)$;\\
& $p.\dstate[c][q.\addr] \Leftarrow q.\myto$; $\deq(\cpReq[c])$;\\
\hline
\hline
\multicolumn{2}{|l|}{\textbf{ChildRecvResp}$(c)$: Child $c$ receiving a response}\\
\hline
Guard: & 
$\avail(\pc[c]) \wedge \mylet r := \first(\pc[c]) \myin r.\typ = \Resp$\\
\hline
Action: & $\myif c.\state[r.\addr] = \In \mythen c.\data[a] \Leftarrow r.\data$;\\
&$\myif c.\waitS[r.\addr] \le r.\myto \mythen c.\wait[a] \Leftarrow \False $;\\
& $c.\state[r.\addr] \Leftarrow r.\myto$; $\deq(\pc[c])$;\\
\hline
\end{tabular}
\caption{Child sending upgrade request and getting back the response}
\label{childside}
\end{subfigure}

\begin{subfigure}{\textwidth}
\centering
\begin{tabular}{|ll|}
\hline
\multicolumn{2}{|l|}{\textbf{ParentSendReq}$(p, c, x, a)$: Parent $p$ sending request to child $c$ to downgrade to $x$ for address $a$}\\
\hline
Guard: & $\parent(c,p) \wedge p.\dstate[c][a] > x \wedge p.\dwait[c][a] = \False$\\
\hline
Action: & $p.\dwaitS[c][a] \Leftarrow x$; $p.\dwait[c][a] \Leftarrow \True$; $\enq(\pc[c], \langle \Req, x, a, \_ \rangle)$;\\
\hline
\hline
\multicolumn{2}{|p{\textwidth}|}{\textbf{ChildRecvReq}$(c)$: Child $c$ receiving a request from its parent and has to downgrade}\\
\hline
Guard: & 
$\avail(\pc[c]) \wedge \mylet q := \first(\pc[c]) \myin q.\typ = \Req \wedge$ \\
& $(\forall i, \parent(i, c) \rightarrow c.\dstate[i][q.\addr] \le q.\myto) \wedge q.\myto < c.\state[q.\addr]$\\
\hline
Action: & $\enq(\cpResp[c], \langle c.\state[q.\addr], q.\myto, q.\addr,$\\
& $\;\;\;%
\myif c.\state[q.\addr] = \Mo \mythen c.\data[q.\addr] \myelse \_\rangle)$;\\
& $c.\state[q.\addr] \Leftarrow q.\myto$; $\deq(\pc[c])$;\\
\hline
\hline
\multicolumn{2}{|l|}{\textbf{ParentRecvResp}$(p, c)$: Parent $p$ receiving a response from child $c$}\\
\hline
Guard: & 
$\parent(c,p) \wedge \avail(\cpResp[c]) \wedge \mylet r := \first(\cpResp[c]) \myin$\\
& $r.\from = p.\state[c][r.\addr]$\\
\hline
Action: & $\myif p.\dstate[c][r.\addr] = \Mo \mythen p.\data[r.\addr] \Leftarrow r.\data$;\\
&$\myif p.\dwaitS[c][r.\addr] \ge r.\myto \mythen p.\dwait[c][r.\addr] \Leftarrow \False $;\\
& $p.\dstate[c][r.\addr] \Leftarrow r.\myto$; $\deq(\cpResp[c])$;\\
\hline
\end{tabular}
\caption{Parent sending downgrade request and getting back the response}
\label{parentside}
\end{subfigure}

\begin{subfigure}{\textwidth}
\centering
\begin{tabular}{|ll|}
\hline
\multicolumn{2}{|p{\textwidth}|}{\textbf{ChildVolResp}$(c, x, a)$: Child $c$ sending a response to downgrade to $x$ for address $a$ without any request from its parent}\\
\hline
Guard: & $(\forall i, \parent(i, c) \rightarrow c.\dstate[i][a] \le x) \wedge x < c.\state[a] \wedge c.\wait[a] = \False$\\
\hline
Action: & $\enq(\cpResp[c], \langle c.\state[a], x, a,%$\\
%& $\;\;\;%
\myif c.\state[a] = \Mo \mythen c.\data[a] \myelse \_\rangle)$;\\
& $c.\state[a] \Leftarrow x$;\\
\hline
\hline
\multicolumn{2}{|p{\textwidth}|}{\textbf{ChildDropReq}$(c)$: Child $c$ receiving a request from its parent and has already downgraded}\\
\hline
Guard: & 
%$\avail(\pc[c]) \wedge \mylet q := \first(\pc[c]) \myin q.\typ = \Req \wedge%$\\
%& $%
%q.\myto \ge c.\state[q.\addr]$\\
$\avail(\pc[c]) \wedge \mylet q := \first(\pc[c]) \myin q.\typ = \Req \wedge q.\myto \ge c.\state[q.\addr]$\\
\hline
Action: & $\deq(\pc[c])$;\\
\hline
\end{tabular}
\caption{Other transitions at a non-root cache}
\label{childextra}
\end{subfigure}

\begin{subfigure}{\textwidth}
\centering
\begin{tabular}{|ll|}
\hline
\multicolumn{2}{|l|}{\textbf{LoadReq}$(c)$: Handling a processor's load request at a leaf cache}\\
\hline
Guard: & $\leaf(c) \wedge \mylet q := \reqFn(c, c.\pos) \myin q.\desc = \Ld \wedge c.\state[q.\addrQ] \ge \Sh$\\
\hline
Action:& $c.\pos \Leftarrow c.\pos + 1$;\\
\hline
\hline
\multicolumn{2}{|l|}{\textbf{StoreReq}$(c)$: Handling a processor's store request at a leaf cache}\\
\hline
Guard: & $\leaf(c) \wedge \mylet q := \reqFn(c, c.\pos) \myin q.\desc = \St \wedge c.\state[q.\addrQ] = \Mo$\\
\hline
\hline
Action:& $c.\pos \Leftarrow c.\pos + 1$; $c.\data[q.\addrQ] \Leftarrow q.\dataQ$;\\
\hline
\end{tabular}
\caption{Handling requests from the processor}
\label{procside}
\end{subfigure}
\caption{Atomic transitions for a cache coherent memory subsystem}
\label{trans}
\end{figure}

Note that the atomic transitions in Figure~\ref{trans} access only local states,
\ie{} a transition can read or write states only corresponding to a single cache
and/or the channel the cache is connected to. This restriction allows such a
system to be implemented directly into hardware. Bluespec System Verilog (BSV),
for instance, directly converts these transitions into efficient synchronous
hardware. Though these transitions logically happen one by one, BSV executes
several transitions simultaneously, as long as they do not access the same
state, \ie{} do not \emph{conflict}~\cite{Hoe:TCAD,HoeArvind:TRSSynthesis1}.
Sometimes, these transitions (both the guards and the actions) may not be
amenable to single cycle implementations in hardware. For example, writing a
register array (like a cache) can actually take several hardware clock cycles,
though logically it is a single transition.  Karczmarek \etal~\cite{Karczmarek}
has provided a scheme to convert atomic transitions into synchronous hardware
in which each transition can span several clock cycles.
