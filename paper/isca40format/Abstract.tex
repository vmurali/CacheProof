We present a method to develop implementations of hierarchical,
message-passing, directory-based, cache-coherence protocols. Our method is
based on the cache system preserving two invariants: a) the directory's notion
of the permissions for the caches it serves is ``conservative'', \ie the caches
do not have more permission than what is assumed by the directory, and b) every
request by a cache or a directory will eventually get a response. We specify an
abstract protocol, that is, a set of meta rules for state transitions, which
has been shown, using the Coq theorem prover, to preserve these invariants. We
have also shown that this abstract protocol can be implemented using $2N + 1$
virtual channels where $N$ is the number of levels and each channel can have as
few as one buffer. 

In this paper, we use the abstract protocol to develop formally proven
implementations of a number of well-known protocols like MSI, MESI, MOSI and
MOESI, both in the presence as well as in the absence of point-to-point FIFO
networks. All these protocols can be expressed in terms of an extra
``compatibility of state'' check in the state transitions allowed by the
abstract protocol. We finally show that the protocols we design using our
methodology have no extra overhead (in terms of state bits or the number of
messages passed to service each request) compared to the existing
implementations of these protocols. 
