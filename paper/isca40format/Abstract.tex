Designing message-passing, directory-based cache coherence protocols has been
very difficult and error-prone because of the following reasons: a) a directory
can not look into the states of other caches to service a request, it has to
rely on the ``correctness'' of the directory's own approximation of other
caches' states, and b) the directory and the caches must ensure that every
request is served in order to avoid deadlocks. One encounters several corner
cases which violate either the ``correctness'' of the directory's states, or
create a scenario where the cache or the directory enters a state which can not
handle a particular (request) message leading to a deadlock. With the increase
in number of cores in modern systems, this protocol design problem gets
exacerbated because of the need for hierarchical caches and the associated
hierarchical directories.

In this paper, we present a systematic methodology to develop hierarchical
message-passing directory-based inclusive cache coherence protocols. We provide
the semantics describing the states in caches and directories, and the
behaviors of the cache and directory controllers which will ensure the above
two properties, \viz that the directory's approximation of its (children)
caches' states is ``correct'' and that every request gets a response. The fact
that these semantics ensure that the two properties hold has been formally
proven using the Coq theorem prover for any number of cache levels and any
number of caches in each level of the hierarchy; the methodology scales without
any additional verification requirement. We demonstrate how a family of
protocols starting from a simple MSI protocol, all the way to
cache-intervention based MOESI protocol can be reduced to the presented
semantics, thus proving the correctness of all these protocols. We finally show
that our methodology, while formally proven, does not impose any additional
requirements in terms of more state bits or passing of more messages than
published protocols, \ie the methodology has no overhead in spite of being
formally proven.
