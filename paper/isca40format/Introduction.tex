\section{Introduction} 

Large shared-memory multicores are likely to use cache-coherence protocols
which are directory-based, hierarchical, and implemented using message passing.
Even though such protocols have been around for two decades (see, for example,
DASH\cite{DASH}, Alewife\cite{Anant}, ...) correct design and implementation of
such protocols remains a challenge. It is difficult to convince oneself that
all corner cases have been handled adequately without a formal proof. And yet
formal proofs of real implementations are practically non-existent. One can't
settle for conservative choices in cache-coherence protocol designs because
performance (\eg the number of messages needed to maintain coherence, the
buffer requirements, etc) is as important as correctness. Consequently proving
a simple protocol for a small number of caches, under unbounded buffer
assumption is of limited use in gaining confidence in the correctness of a real
protocol. 

In this paper we will consider systems where caches form a tree hierarchy and
caches are \emph{inclusive}, that is, if a cache contains address $a$ then its
parent is guaranteed to contain address $a$, though the parent may have stale
data for $a$. Every cache or node in the system has an associated directory
which contains information about its children caches. The coherence state
changes are orchestrated by passing requests and responses between the caches
via the directory.  Unlike bus-based protocols which lock the bus till all the
required state transitions are done thereby ensuring atomicity of state
transitions, in a message-passing protocol, requests and responses are split
leading to transient states. This creates scenarios not present in bus-based
protocols, which can lead to coherence violations or deadlocks, if the protocol
is not designed or implemented carefully.

Several factors contribute to the difficulty of designing protocols for such
distributed, hierarchical and message-passing systems:

\begin{enumerate}
\item Lots of concurrent events must be dealt with without a central authority
to impose a sequence on events;
\item The state on which transitions are based is physically distributed in
space and cannot be accessed atomically;
\item Physical network may block the transmission of a message because of lack
of communication buffers or head-of-the-line blocking; 
\item The cache or directory may reach a state which can not handle a
particular message. An incomplete set of transition rules may cause the
protocol to deadlock.
\end{enumerate}

As examples of problems in designing correct protocols, we describe several
concrete and familiar scenarios from two-level systems (a private L1 cache for
each processor and a memory with a directory):

\noindent \emph{Example 1:} A cache in state $S$ for a line gets a store request
from the processor and sends a request for changing the line's state to $M$ to
the directory controller. The controller may eventually send a response
permitting the cache to go to $M$ state for that line. The response would
signify that the states in the other caches for that line \emph{were} $I$. But
the messages take time to propagate and the response is necessarily based on the
old states of other caches -- the current state of another cache may not be $I$,
violating the Single-writer-at-a-time invariant and hence breaking cache
coherence.

\noindent \emph{Example 2:} A cache in state $S$ for a line gets a store request
from the processor and sends a request for changing the line's state to $M$ to
the directory controller. Instead of getting a response permitting it to go to
$M$ state, the cache may receive a request from the directory controller asking
this cache to go to $I$ state (invalidate request)-- this could have happened if
the directory is serving some other cache's request to go to $M$ state for the
same line. It is not easy to determine the correct behavior of the former cache
on receiving the invalidate request -- whether it should ignore the invalidate
request and wait for a response to its request or whether it should invalidate
the line. Would ignoring the invalidate request lead to a deadlock?

\noindent \emph{Example 3:} A cache receives an invalidate request from the
directory controller for a line that is not present in the cache. Is such a
request even possible? The directory does not have access to the up-to-date
state of the cache, and so can potentially make this request. How should a
cache handle this request; should it drop such a request, or respond saying
that it does not have the line?

These examples give a glimpse of the complexity of designing and implementing a
correct cache coherence protocol in a directory-based non-bus setting.  Even for
a 2-level cache hierarchy, designing and implementing the protocol is quite
difficult. This problem is greatly exacerbated in a multi-level cache hierarchy,
which is becoming the norm because of the increasing core counts. 

In this paper, we present a substantially easier method to design
directory-based cache-coherence protocols for such systems. The method breaks
down the design problem in two parts: a part that is shared by a number of
concrete protocols which we call \glob{}s, and another part that is specific to
the protocol to be designed and implemented, called \policy{}. \glob{}s provide
guarantees such as every request will eventually get a response, the
directory's notion of the cache states is ``conservative'' \ie a cache can not
have more permissions for a line than what the directory thinks it has, \etc;
\glob{} enables \policy{} to focus on the high level properties such as the
set of actions a cache or a directory should perform to maintain the
Single-writer-at-a-time and the Read-from-last-writer invariants.

The \glob{}s can in turn be guaranteed using a set of (\policy{} independent)
\local{}s which dictate local properties of a node such as when a request can be
handled, what state changes happen on sending a response, properties about the network,
\etc. We have formally proved that \local{}s ensure \glob{}s using the
Coq Theorem Prover \cite{}, however, a presentation of the formal proof is
beyond the scope of this paper. The \glob{}s and the \local{}s form the
cornerstone of our framework. The separation of \glob{}s and \local{}s is akin
to having an abstraction layer which hides low level details.

The \glob{}s of our framework can be used to design a wide range of cache
coherence protocols, notably MSI, MESI, MOSI, MOESI, as well as their
hierarchical version. We will show how these are designed as \policy{}s,
focusing on high-level details.

Our method substantially reduces the design and proof burden for new protocols
because of the common abstract protocol. The abstract protocol is a set of
transitions that the caches and directories follow to send or receive a request
or a response. We have formally proved that these transitions ensure the
following properties: a) the directory's notion of a cache's state is
``conservative'' in that a cache is never going to have more permissions that
what the directory thinks the cache has; and b) any request will eventually get
a corresponding response. A formal proof of the abstract protocol was written
using the Coq theorem prover \cite{}, however, a presentation of the formal
proof is beyond the scope of this paper.

It should be noted that the flexibility of our methodology to design multiple
cache coherence protocols comes from the fact that we do not assign any meaning
to the value of a coherence state beyond upgrades or downgrades. Thus, how or
why a particular cache state implies that a particular address may not be
present in any other cache stems from protocol specific ``compatibility of
state'' check. 

The main contributions of this paper are: 1. A new abstract protocol that
satisfies two important invariants for distributed and hierarchal
message-passing protocols; 2. Instantiation of the abstract protocol to
implement four well-known protocols (MSI, MESI, MOSI, MOESI) and their
variations and showing why their correctness is guaranteed as long as the
abstract protocol is correct, \ie it provably implements the two invariants;
and 3. Our method does not introduce any overhead over direct implementation of
these protocols.  We think that this new method of designing protocols
substantially reduces the required effort, and that our abstract protocol is
flexible enough to implement (while ensuring correctness of) most if not all
inclusive, message-passing, directory-based protocols.

\paragraph{Paper organization:} Section \ref{sec:abstract} describes the abstract
protocol; Section \ref{sec:properties} gives the two properties that are
guaranteed by the abstract protocol and provides some intuition for proving
these properties. We also discuss the communication buffer requirements for the
abstract protocol and show that the requirement is only proportional to the
number of levels and not to the number of processors. In Section
\ref{sec:protocols}, we show how the abstract protocol can be used to implement
MSI, MESI, MOSI and MOESI protocols and their variants. In Section
\ref{sec:performance}, we show that these formally verified family of protocols
require no extra state bits, nor extra message transfers compared to other
published implementations. In Section \ref{sec:related}, we present some of the
related work and finally in Section \ref{sec:conclusion}, we present some
directions for the future work.
