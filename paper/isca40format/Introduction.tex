\section{Introduction} 

Large shared-memory multicores are likely to use cache-coherence protocols
which are directory-based, hierarchical, and implemented using message passing.
Even though such protocols have been around for two decades (see, for example,
DASH\cite{DASH}, Alewife\cite{Anant}, ...) correct design and implementation of
such protocols remains a challenge. It is difficult to convince oneself that
all corner cases have been handled adequately without a formal proof. And yet
formal proofs of real implementations are practically non-existent. One can't
settle for conservative choices in cache-coherence protocol designs because
performance (\eg the number of messages needed to maintain coherence, the
buffer requirements, etc) is as important as correctness. Consequently proving
a simple protocol for a small number of caches, under unbounded buffer
assumption is of limited use in gaining confidence in the correctness of a real
protocol. 

In this paper we will consider systems where caches form a tree hierarchy and
caches are \emph{inclusive}, that is, if a cache contains address $a$ then its
parent is guaranteed to contain address $a$, though the parent may have stale
data for $a$. Every cache or node in the system has an associated directory
which contains information about its children caches. The coherence state
changes are orchestrated by passing requests and responses between the caches
via the directory.  Unlike bus-based protocols which lock the bus till all the
required state transitions are done thereby ensuring atomicity of state
transitions, in a message-passing protocol, requests and responses are split
leading to transient states. This creates scenarios not present in bus-based
protocols, which can lead to coherence violations or deadlocks, if the protocol
is not designed or implemented carefully.

Several factors contribute to the difficulty of designing protocols for such
distributed, hierarchical and message-passing systems:

\begin{enumerate} \item Lots of concurrent events must be dealt with without a
central authority to impose a sequence on events; \item The state on which
transitions are based is physically distributed in space and cannot be accessed
atomically; \item Physical network may block the transmission of a message
because of lack of communication buffers or head-of-the-line blocking; \item
The cache or directory may reach a state which can not handle a particular
message. An incomplete set of transition rules may cause the protocol to
deadlock.  \end{enumerate}

As examples of problems in designing correct protocols, we describe several
concrete and familiar scenarios from two-level systems (a private L1 cache for
each processor and a memory with a directory). States $M, S$ and $I$ have their
usual meaning in MSI protocol ($M$ represents read and write permissions for an
address, $S$, read-only permissions and $I$, no permission)

\noindent \emph{Example 1:} A cache in state $S$ for a line gets a store
request from the processor and sends a request for changing the line's state to
$M$ to the directory controller. The controller may eventually send a response
permitting the cache to go to $M$ state for that line. The response would
signify that the states in the other caches for that line \emph{were} $I$. But
the messages take time to propagate and the response is necessarily based on
the old states of other caches -- the current state of another cache may not be
$I$, violating the Single-writer-at-a-time invariant and hence breaking cache
coherence.

\noindent \emph{Example 2:} A cache in state $S$ for a line gets a store
request from the processor and sends a request for changing the line's state to
$M$ to the directory controller. Instead of getting a response permitting it to
go to $M$ state, the cache may receive a request from the directory controller
asking this cache to go to $I$ state (invalidate request)-- this could have
happened if the directory is serving some other cache's request to go to $M$
state for the same line. It is not easy to determine the correct behavior of
the former cache on receiving the invalidate request -- whether it should
ignore the invalidate request and wait for a response to its request or whether
it should invalidate the line. Would ignoring the invalidate request lead to a
deadlock?

\noindent \emph{Example 3:} A cache receives an invalidate request from the
directory controller for a line that is not present in the cache. Is such a
request even possible? The directory does not have access to the up-to-date
state of the cache, and so can potentially make this request. How should a
cache handle this request; should it drop such a request, or respond saying
that it does not have the line?

These examples give a glimpse of the complexity of designing and implementing a
correct cache coherence protocol in a directory-based non-bus setting.  Even
for a 2-level cache hierarchy, designing and implementing the protocol is quite
difficult. This problem is greatly exacerbated in a multi-level cache
hierarchy, which is becoming the norm because of the increasing core counts. 


In this paper, we present a substantially easier method to design cache
coherence protocols for such systems. The user specifies a protocol for a
two-level system without message passing. The protocol rules can simultaneously
or \emph{atomically} examine and change the state of all the caches in the
system. For example, the MSI protocol may be described as follows: 

***** Murli insert the description of MSI **** We give a procedure that
transforms such simple and global protocol descriptions into a distributed,
message-passing protocol for hierarchal caches. Our procedure is based on
preserving some invariants in rule transformations and has been shown to be
sound and complete with respect the global description using the coq theorem
prover \cite{}, however, a discussion of the formal proof is beyond the scope
of this paper. Both the procedure and the proof are parameterized by the number
of cache levels and the number of caches in each level of the hierarchy as well
as by a predicate which involves a local ``compatibility of state'' test. 

It should be noted that the flexibility of our procedure comes from the fact
that we do not assign any meaning to the value of a coherence state beyond
upgrades or downgrades. Thus, how or why a particular cache state implies that
a particular address may not be present in any other cache stems from protocol
specific ``compatibility of state'' check. 

The main contributions of this paper are: 1. An automatic procedure to refine
centralized protocols without message passing into protocols for distributed,
hierarchal, cache system with message-passing; 2. Specification of the
invariants to be preserved by the procedure; and 3. Showing that our procedure
does not introduce any overhead over direct implementation of these protocols.
We think that our method of designing protocols substantially reduces the
required effort, and that our procedure is flexible enough to implement (while
ensuring correctness of) most, if not all, inclusive message-passing
directory-based protocols.

\paragraph{Paper organization:} *** to be redone *** Section \ref{sec:abstract}
describes the abstract protocol; Section \ref{sec:properties} gives the two
properties that are guaranteed by the abstract protocol and provides some
intuition for proving these properties. We also discuss the communication
buffer requirements for the abstract protocol and show that the requirement is
only proportional to the number of levels and not to the number of processors.
In Section \ref{sec:protocols}, we show how the abstract protocol can be used
to implement MSI, MESI, MOSI and MOESI protocols and their variants.  In
Section \ref{sec:performance}, we show that these formally verified family of
protocols require no extra state bits, nor extra message transfers compared to
other published implementations.  In Section \ref{sec:related}, we present some
of the related work and finally in Section \ref{sec:conclusion}, we present
some directions for the future work.
