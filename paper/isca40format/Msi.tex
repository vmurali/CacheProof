\section{A two-level message-passing directory-based MSI protocol}
\label{sec:msi}

\newcommand{\fourAngle}[6]{\text{$#1\langle#2\rightarrow#3,#4,#5\rightarrow#6\rangle$}}
\newcommand{\threeAngle}[5]{\text{$#1\langle#2\rightarrow#3,#4,#5\rangle$}}

%\newcommand{\Req}[5]{\fourAngle{Req}{#1}{#2}{#3}{#4}{#5}}
\newcommand{\Req}[4]{\threeAngle{Req}{#1}{#2}{#3}{#4}}
\newcommand{\Resp}[4]{\threeAngle{Resp}{#1}{#2}{#3}{#4}}

\newcommand{\msg}[5]{\text{$#1 \langle #2 \rightarrow #3, #4, #5 \rangle$}}
%\newcommand{\Req}[5]{\msg{Req}{#1}{#2}{#3}{#4}{#5}}
%\newcommand{\Resp}[5]{\msg{Resp}{#1}{#2}{#3}{#4}{#5}}
\newcommand{\Data}[4]{\msg{Data}{#1}{#2}{#3}{#4}}
\newcommand{\Mesg}[4]{\msg{Mesg}{#1}{#2}{#3}{#4}}
\newcommand{\DirInv}{\emph{DirInv}}
%\newcommand{\fourAngle}[5]{\text{$#1\langle#2,#3,#4,#5\rangle$}}
\newcommand{\ReqCore}[4]{\fourAngle{Req}{#1}{#2}{#3}{#4}}

\subsection{Atomic MSI protocol for 2-levels, no replacement evictions}

XXXXXX: This subsection is probably not going to go into the paper, since it's
too obvious. The definitions have been repeated in the later subsections.

We start with an atomic MSI protocol, with two levels of the memory hierarchy --
the L1 caches and the main memory. In this atomic protocol, the memory
controller can read and write the coherence states as well as the data of each
address present in the L1 caches. As usual, coherence state for an address in
the cache denotes the permissions that the cache has for the address: $M = \{R,
W\}, S = \{R\}, I = \{\}$.

Let us assume that the caches have enough capacity so that no line gets evicted
for the sake of replacement. Lines may however get evicted to enforce the
single-writer property, \ie only one L1 cache can modify a cache line at any
time. Figure~\ref{alg:msi-easy} shows the pseudo-code for the behaviors of the
memory controller and each of the caches to implement the MSI protocol, using
the notations given in Figure~\ref{fig:notation}. 

\begin{figure}
\begin{figure}[H]
\begin{boxedminipage}{\linewidth}
\begin{algorithmic}
\State Pick a request \ReqCore{c}{a}{x}{d} from core to cache $c$ \newline
$s.t.$ $state[c][a] < x$;
\ForAll {$c' \neq c,\; s.t.\; toCompatible(x) < state[c'][a]$}
  \If {$state[c'][a] = M$}
    \State $memory[a] \gets data[c'][a]$;
  \EndIf
  \State $state[c'][a] \gets toCompatible(x)$;
\EndFor
\If {$state[c][a] = I$}
  \State $data[c][a] \gets memory[a]$;
\EndIf
\State $state[c][a] \gets x$;
\end{algorithmic}
\end{boxedminipage}
\subcaption{Behavior of memory controller}
\label{alg:mem}
\end{figure}
\begin{figure}[H]
\begin{boxedminipage}{\linewidth}
\begin{algorithmic}
\State Pick a request \ReqCore{c}{a}{x}{d} from core to cache $c$ \newline
$s.t.$ $state[c][a] \not< x$;
\If {$state[c][a] = M$}
  \State $update(c, a, d)$;
\ElsIf {$state[c][a] = S$}
  \State $respond(c, a, d)$;
\EndIf
\State remove \ReqCore{c}{a}{x}{d}
\end{algorithmic}
\end{boxedminipage}
\subcaption{Behavior of cache $c$}
\end{figure}
\caption{Atomic 2-level MSI protocol: memory controller has instantaneous
access to all states and data of all L1 caches and data in the memory}
\label{alg:msi-easy}
\end{figure}

\begin{figure}
\begin{minipage}{\linewidth}
\begin{tabularx}{\linewidth}{|cX|}
\hline
$toCompatible(s)$ & If the state of one cache is $s$ for address $a$, the
highest state of another cache for the same $a$
$\newline
toCompatible(M) = I,\newline
toCompatible(S) = S,\newline
toCompatible(I) = M
$\\
\hline
$x < y$ & Does state $x$ have fewer permissions that state $y$?
$I < M, I < S, S < M$\\
\hline
$\ReqCore{c}{a}{x}{d}$ & The incoming request for cache $c$ from a core\\
\multicolumn{1}{|r}{$a$} & the address of the cache line\\
\multicolumn{1}{|r}{$x$} & the state of the cache line the request is expecting
(\ie the minimum permissions required)\\
\multicolumn{1}{|r}{$d$} & information about the word offset, the byte enables
and the data (only in the case of store)\\
\hline
$update(c, a, d)$ & Action denoting update of a cache line for address $a$ in
cache $c$ with the incoming request's information $d$\\
\hline
$respond(c, a, d)$ & Action denoting cache $c$ responding to a read request for
address $a$ with the incoming request's information $d$\\
\hline
\end{tabularx}
\subcaption{Abstraction functions}
\end{minipage}

\begin{minipage}{\linewidth}
\begin{tabularx}{\linewidth}{|cX|}
\hline
$state[c][a]$ & coherence state (from set $\{M, S, I\}$) of cache $c$ for
address $a$\\
$data[c][a]$ & data line of cache $c$ for address $a$\\
$memory[a]$ & data line of memory for address $a$\\
\hline
\end{tabularx}
\subcaption{Values stored by a cache for an address}
\end{minipage}
\caption{Notations used in Figure~\ref{alg:msi-easy}}
\label{fig:notation}
\end{figure}

XXXXXX: End of obvious subsection


Let us walk through our framework using the example of an MSI protocol.  We will
start with a simple system for the sake of explanation.  The system has two
levels of memory hierarchy -- private L1 caches for each core and a shared
memory. The L1 caches do not have direct access to the memory, and the memory
does not have a direct access to the actual states in the  L1 caches, or L1's
data. Instead, the main memory has direct access to a directory associated with
it which keeps a version of the L1 cache states for each address. The L1 caches
do not have direct access to this directory. The directory's version of the L1
cache states may differ from the actual L1 cache states. We use the terms memory
and directory interchangeably.

\begin{figure}
\begin{tabularx}{\linewidth}{|cX|}
\hline
$state[c][a]$ & coherence state (from set $\{M, S, I\}$) of cache $c$ for address $a$\\
$data[c][a]$ & data in cache $c$ for address $a$\\
\hline \end{tabularx}
\caption{Values stored by an L1 cache $c$ for an address $a$}
\label{table:lineinfo}
\end{figure}

\begin{wrapfigure}{r}{.4\linewidth}
\centering
\begin{tabular}{|c|ccc|}
\hline
& $M$ & $S$ & $I$\\
\hline
$M$ & $=$ & $>$ & $>$\\
$S$ & $<$ & $=$ & $>$\\
$I$ & $<$ & $<$ & $=$\\
\hline
\end{tabular}
\caption{$<$ and other relations for MSI states}
\label{msi<}
\end{wrapfigure}

Each L1 cache stores the information given in Figure \ref{table:lineinfo}. The
coherence state for an address denotes the permission that a cache has for the
address: $M = \{R, W\}, S = \{R\}, I = \{\}$. We can define a $<$ relation for
coherence state values based on the permissions the states represent. This is
shown in Figure \ref{msi<}. State $I$ is akin to the cache not having the data
corresponding to $a$, since it can neither read or write the $a$'s data. Figure
\ref{table:dirinfo} shows the information stored in the memory and directory. 

\begin{figure}
\begin{tabularx}{\linewidth}{|cX|}
\hline
$dir[c][a]$ & directory's version of the coherence state of cache $c$ for address $a$\\
$memory[a]$ & data in the memory address $a$\\
\hline
\end{tabularx}
\caption{Values stored by a directory and memory for an address}
\label{table:dirinfo}
\end{figure}

The L1 caches and the directory communicate with each using via messages. Each
L1 cache communicates with the memory directly via 4 buffers -- two incoming
buffers from the memory, one each for requests and responses from the memory,
and two outgoing buffers into the memory, one each for requests and responses to
the memory. All these buffers converge into or diverge from 4 buffers of the
memory -- two incoming buffers, one each for requests and responses from the
children and two outgoing buffers, one each for requests and responses to the
children (Figure \ref{fig:setup}). All these buffers have the FIFO property. Not
only that, we will also assume that requests sent from a source will never reach
its destination till an earlier response sent by the same source has been
handled (and hence removed) by the destination. We will show in Section
\ref{sec:fifo} about how these network requirements can be realistically
implemented.

\subsection{No replacement evictions}

We will first design an MSI protocol in which the L1 caches have enough
capacity (or rather associativity) so that no line gets evicted due to
replacement. Lines may however get evicted to enforce the following invariant
which is needed for maintaining cache coherence.

\begin{inv}
Single-writer: When a cache has permission to modify data for an address ($M$
state), no other cache has read or write permission for that address. In other
words, the other caches must be in $I$ state.
\label{singleWriter}
\end{inv}

\begin{wrapfigure}{r}{.46\linewidth}
\begin{subfigure}{.27\linewidth}
\centering
\begin{tabular}{|ccc|}
\hline
$M$ & $\rightarrow$ & $I$\\
$S$ & $\rightarrow$ & $S$\\
$I$ & $\rightarrow$ & $I$\\
\hline
\end{tabular}
\caption*{$toCompatible$}
\label{toCompat}
\end{subfigure}
~~~~~~~~~~~~~~
\begin{subfigure}{.21\linewidth}
\centering
\begin{tabular}{|c|c|}
\hline
$M$ & $\checkmark$\\
$S$ & $\times$\\
$I$ & $\times$\\
\hline
\end{tabular}
\caption*{$isModified$}
\label{isModified}
\end{subfigure}
\caption{$toCompatible$ and $isModified$ mappings}
\label{fig:msimap}
\end{wrapfigure}

Because of the Single-writer invariant, it makes sense to talk about
compatibility of the states of two caches for an address. Two cache states $x$
and $Y$ are \emph{compatible} if two caches having the same address in states
$x$ and $y$ do not violate the Single-writer invariant. We define a mapping
$toCompatible(x)$ which gives the highest state which is compatible with a given
state $x$ (Figure \ref{fig:msimap}).

In addition to Single-writer Invariant, maintaining cache coherence also
requires the following invariant:

\begin{inv}
Read-from-last-writer: When a cache reads data for an address, it reads the last
modified version of the data for the address. If the data for an address has not
been modified, the cache reads the data from memory.
\end{inv}

Function $isModified(x)$ determines if data for an address state $x$ can
potentially be modified or not (Figure \ref{fig:msimap}).

If a cache $c$ receives a request for an address $a$ from the processor, it can
service the request straightaway if it has enough permissions for address $a$.
Otherwise, it has to get to a state for $a$, say $y$, which has enough
permissions. In other words, it has to \emph{upgrade} its state to $y$. In order
to maintain the Single-writer Invariant \ref{singleWriter}, cache $c$ has to
ensure that the states of other caches are \emph{compatible} with $y$. But the
caveat is that $c$ can access only its own state and data, as opposed to that of
other caches (or the directory).

Cache $c$ has to instead communicate with the directory to ensure that the
states of other caches are compatible. $c$ must send a message
\Req{c}{Dir}{a}{y} requesting the directory to tell if it can upgrade its state
for address $a$ to $y$. We call this an \emph{upgrade request}. Let us now take
for granted that such a message can be sent (and see how this in ensured later).

\begin{inv}
A cache will eventually be able to send an upgrade request.
\label{csendreq}
\end{inv}

We will also assume that that a cache can upgrade its state to $y$ only after it
receives a message from the directory \Resp{Dir}{c}{a}{y} permitting it to
upgrade its state for address $a$ to $y$. We call such a message as an
\emph{upgrade response}.

\begin{inv}
Cache $c$ can upgrade $state[c][a]$ from $x$ to $y$ where $y > x$ only after it
receives an upgrade response \Resp{Dir}{c}{a}{y} from the directory.
\label{cwaitresp}
\end{inv}

We introduce two other kinds of messages: \emph{downgrade request}
\Req{Dir}{c'}{a}{y} sent from the directory to a cache $c'$ asking $c'$ to
downgrade its state for address $a$ to $y$, and \emph{downgrade response}
\Resp{c'}{Dir}{a}{y} sent from $c'$ to the directory informing the directory
that $c'$ has changed its state $state[c'][a]$ from $x$ to $y$, where $x > y$.

How can the behavior of the directory be designed (in terms of sending and
receiving messages) to ensure that cache $c$ will get an upgrade response
eventually. Let's say the invariants given in Figure \ref{DirInv} are
somehow guaranteed. With these invariants, let us design the behavior of the
directory.

\floatstyle{boxed}
\restylefloat{figure}
\begin{figure}
\begin{inv}
Just before receiving an upgrade request \Req{c}{Dir}{a}{y} from $c$ $dir[c][a]
< y$.
\label{drecvreq}
\end{inv}
\begin{inv}
The directory will eventually be able to send any downgrade request to any cache
$c$.
\label{dsendreq}
\end{inv}
\begin{inv}
The directory can change $dir[c][a]$ from $x$ to $y$ where $y < x$, \ie
downgrade its directory state only after receiving a downgrade response
\Resp{c}{Dir}{a}{y} from $c$.
\label{dwaitresp}
\end{inv}
\begin{inv}
If the directory has sent a downgrade request \Req{Dir}{c}{a}{y} to cache $c$,
it will eventually get a downgrade response \Resp{Dir}{c}{a}{z} from cache $c$
such that $z \le y$.
\label{drecvresp}
\end{inv}
\begin{inv}
Just before receiving a downgrade response from $c$, if $isModified(dir[c][a])$
is true, then $c$ would have sent the data along with the downgrade response.
\label{drecvdata}
\end{inv}
\begin{inv}
The directory will eventually be able to send any upgrade response to any cache
$c$. It can also send the data to cache $c$ along with the upgrade response.
\label{dsendresp}
\end{inv}
\begin{inv}
Conservative directory:
\begin{spacing}{.1}
\begin{equation*}
\forall c, a,\; dir[c][a] \ge state[c][a]
\end{equation*}
\end{spacing}
\label{conservative}
\end{inv}
\caption{Directory invariants}
\label{DirInv}
\end{figure}
\floatstyle{plain}
\restylefloat{figure}

%The directory ensures that the states of any two caches for an address are
%compatible by the following behavior.
%When the directory receives an upgrade request
%\Req{c}{Dir}{a}{x}, Invariant \ref{drecvreq} guarantees that $dir[c][a] < x$ and
%because of Invariant \ref{conservative}, $cache[c][a] le dir[c][a]$, resulting
%in $cache[c][a] < x$. Unless the directory sends an upgrade response to $c$
%permitting it to upgrade to $x$, $c$ will remain in state $cache[c][a]$ forever
On receiving an upgrade request \Req{c}{Dir}{a}{x}, the directory sends a
downgrade request \Req{Dir}{c'}{a}{toCompatible(x)} to every cache $c'$
for which $dir[c'][a] > toCompatible(x)$. Such a request can be sent eventually
because of Invariant \ref{dsendreq}. The incoming upgrade request is not
removed and no further upgrades requests are processed till an appropriate
upgrade response is sent back from the directory to $c$.

Because of Invariant \ref{drecvresp}, the directory will eventually receive
downgrade responses \Resp{c'}{Dir}{a}{z} from each of the requested caches $c'$
such that $z \le toCompatible(x)$ (since the directory made a downgrade request
to go to $toCompatible(x)$). Just before receiving the downgrade response, if
$isModified(dir[c'][a])$ is true (\ie $dir[c'][a] = M$), then by Invariant
\ref{drecvdata}, the directory would also receive data from $c'$ and the memory
for address $a$ is updated with the received data.

Once all the downgrade responses have been received, since no upgrade responses
have been sent to any cache meanwhile, we have $\forall c' \neq c, dir[c'][a] \le
toCompatible(x)$. Because of Invariant \ref{conservative}, it is guaranteed that
$\forall c' \neq, cache[c'][a] \le toCompatible(x)$. The directory now changes
$dir[c][a]$ to $x$ and sends an upgrade response \Resp{Dir}{c}{a}{x} to $c$. If
$dir[c][a] = I$ right before the upgrade response is sent, then it implies that
$cache[c][a] = I$. So the data from the memory for address $a$ is also sent
along with the upgrade response. Note that such an upgrade response, along with
the data, can always be sent eventually by the directory because of Invariant
\ref{dsendresp}. Finally, the upgrade response from $c$ is removed by the
directory.

The above behavior for the directory ensures that the states of any two caches
for every address are compatible, as can seen by induction.  This property is
true initially because every cache is in state $I$ for every address. If this
property is true before a request is started received, it will be true by the
time the request is removed and an upgrade response is sent. This is because an
upgrade response is sent only when every cache other than the one requesting is
at a state $\le toCompatible(x)$, where $x$ is the state the requesting cache
wants to go to, and $toCompatible(x)$ is the highest state that a cache can be
in to be compatible with another cache in state $x$ for an address.

Whenever the directory receives a downgrade response from a cache $c'$, just
before receiving the response if $isModified(dir[c'][a])$ is true, \ie
$dir[c'][a] = M$, $c'$ must have been the only cache with permission to modify
address $a$. This is because the directory ensures that the states of any two
caches for address $a$ is compatible at all times (as shown above). Thus, to
ensure the Read-from-last-writer Invariant, the potentially modified data from
$c'$ is written back from into the memory on receiving the downgrade response
from $c'$ as seen in our design.

The invariants \ref{drecvreq} to \ref{conservative} in Figure \ref{DirInv}
essentially makes it straight forward to design the behavior of the directory.
For the sake of argument, let us assume Invariant \ref{drecvreq} not be
guaranteed. Consider a scenario in which the cache $c$ which has sent a request
\Req{c}{Dir}{a}{M}. Before receiving an upgrade response for this request, $c$
sends another upgrade request \Req{c}{Dir}{a}{S}. Let us assume that the
directory processed the requests in the order in which they were sent. Once the
directory has finished processing the first request, it changes its directory
state $dir[c][a]$ to $M$. When it receives the second request, it is not easy to
decide if the directory has to drop the request or to send another response
authorizing the cache to change to $S$. But if Invariant \ref{drecvreq} is
guaranteed, it's guaranteed that such a scenario will never arise.

The difficulty in message passing protocols comes from the fact that it is
difficult to guarantee the above invariants, even for a 2-level memory hierarchy
discussed above as shown in the following scenarios.

%\begin{scen}
%\emph{Inability to send messages because of network congestion:}
%The ability of a cache or the directory to send messages (Invariants
%\ref{dsendreq}, \ref{dsendresp} and \ref{csendreq}) depends upon the
%availability of free buffers in the network; if the network remains congested
%all the time, then it would not be possible to send the messages, and the system
%would deadlock.
%\end{scen}
%
%\begin{scen}
%\emph{Multiple upgrade requests from the cache for the same address:}
%If a cache is allowed to make multiple requests to upgrade the state of an
%address (say to $M$ and then to $S$) before receiving the upgrade responses,
%then it might violate Invariant \ref{drecvreq} if the requests were processed by
%the directory in the order they were received. This makes the decision on
%whether the directory drops the second request or not hard to reason about, as
%shown above.
%\end{scen}

\begin{scen}
\emph{On sending an upgrade request, the cache stops processing downgrade
requests it gets an upgrade response:}
Consider a scenario with two caches $c$ and $c'$, with both the caches in state
$S$ for address $a$. Both the caches get a store request from the processor and
send \Req{c}{Dir}{a}{M} and \Req{c'}{Dir}{a}{M}, respectively, to the directory.
Let's say \Req{c'}{Dir}{a}{M} arrives at the directory first. The directory
sends a downgrade request to cache $c$, but since $c$ does not process any
downgrade requests till it gets its upgrade response, it will never send back a
downgrade response to the directory, violating Invariant \ref{drecvresp},
resulting in a deadlock. Note that this deadlock would occur even if a cache is
waiting for the upgrade response, and stops processing downgrade requests for
any address, even if the address corresponding to the downgrade request is not
the same as the one corresponding to the upgrade response.
\end{scen}

\begin{scen}
\emph{At the directory, downgrade responses from children are blocked by upgrade
requests:}
Consider a scenario where two caches $c$ and $c'$ are both in state $S$ for
address $a$. $c'$ gets a store request from the processor and sends an upgrade
request \Req{c'}{Dir}{a}{M} to the directory. The directory receives that
request, and sends a downgrade request \Req{c}{Dir}{a}{I} to $c$. $c$ sends a
downgrade authorization \Resp{c}{Dir}{a}{I} to the directory on receiving the
downgrade request. But the directory can not receive the authorization because
it is blocked by other requests (including \Req{c'}{Dir}{a}{M}). This violates
Invariant \ref{drecvresp} and hence creates a deadlock.
\end{scen}

These scenarios show that if the invariants of Figure \ref{DirInv} are violated,
then it either breaks cache coherence (by creating deadlocks) or it makes
it hard to reason about the behavior of the directory.

Before we show how the invariants of Figure \ref{DirInv} can be guaranteed, let
us consider a slightly more complicated system where caches can evict lines for
replacement.

\subsection{Replacement evictions are allowed}
If a cache receives a request from the core for an address $a$ and has no space
to accommodate address $a$, it evicts another address $a'$ to make room for $a$.

In the previous scenario, the caches didn't have to evict for replacement; a
cache had to send downgrade responses to the directory only on receiving a
corresponding request. But in the new scenario, a cache can send a downgrade
notification voluntarily to the directory on evicting an address. Such a
notification is useful in keeping the directory's version of the cache's state
up to date. We will restrict the family of directory-based protocols that that
we discuss to those in which the caches send such notifications.

Since the directory can not distinguish between the downgrade notification sent
voluntarily and a downgrade response sent because of a downgrade request from
the directory, we will use the same message to denote both.

We will first discuss an example of the behavior of the L1 cache that would
violate Invariant \ref{conservative}.

\begin{scen}
\emph{Cache sends voluntary downgrade responses indiscriminately:}
Consider a system with only two caches $c$ and $c'$ (Figure \ref{fig:cross})
such that $state[c][a] = dir[c][a] = S$, and $state[c'][a] = dir[c'][a] = I$.
Let $c$ be a non-blocking cache, so it can process requests from the core even
if there are pending misses. $c$ receives a store request for address $a$ from
the core and sends \Req{c}{Dir}{a}{M} to the directory.  Before it gets back an
upgrade response from the directory, it receives another load request for
address $a'$ from the core. Suppose the line for $a'$ is not present in $c$ and
$c$ has no space for $a'$, it decides to evict address $a$ to make room for
$a'$. It voluntarily sends a downgrade response \Resp{c}{Dir}{a}{I} to the
directory. Meanwhile, the directory receives the upgrade request
\Req{c}{Dir}{a}{M}, and since the directory state of the only other cache is $I$
for address $a$, the directory sends back an upgrade response. The directory
then receives the downgrade response \Resp{c}{Dir}{a}{I} and changes $dir[c][a]$
to $I$. $c$ finally receives the upgrade response \Resp{Dir}{c}{a}{M} and
upgrades $state[c][a]$ to $M$. This breaks Invariant \ref{conservative}, since
$state[c][a] = M$ and $dir[c][a] = I$. If the directory gets an upgrade to $M$
request for address $a$ from $c'$, it will assume that $c$ is in state $I$ for
address $a$ and send an upgrade to $M$ response to $c'$. This will violate the
Single-writer invariant, thereby violating coherence.
\end{scen}
