\section{Related work}
\label{sec:related}

The work on Manager-Client Pairing \cite{mcp} is closest to our work. It
presents a unifying methodology for designing multi-tier coherence protocols by
defining the interactions between levels within a coherence hierarchy to enable
composition. They define interfaces which stand-alone coherence protocols can
adhere to maintain coherence. We differ from their work because we generate
the whole coherence protocol automatically.

Token coherence \cite{token} proposes an alternative framework to build
coherence protocols, by separating performance from correctness. This avoids
races in the protocol by having a slow correctness substrate-basedd fall-back
mechanism in case of a memory race (detected by a time out).

Model-checking based techniques have been used before in trying to prove cache
coherence protocols \cite{Clarke}, \cite{Chou}, \cite{ip}, \cite{McMillan},
\cite{pong}, \cite{Stern}. They try to tackle model-checking's state explosion
problem using innovative techniques, but the fundamental issue remains -- large
systems can not be verified by model-checking.

Zhang \etal proposed Fractal Coherence \cite{Zhang} for verifying protocols
using model-checking. A minimum system is verified and the protocol has to be
designed in such a way that if a subtree is replaced by a single node, it has
the same interface behavior (the fractal property). The difficulty in this
approach is that fractal property is not common in hierarchical protocols, extra
non-intuitive messages have to be passed around to ensure the property.

There has also been some work on using theorem proving techniques for proving
the correctness of directory-based cache coherence protocols \cite{loew},
\cite{park}. They rely on using abstraction functions to simplify the proof of
coherence protocols, which is similar to our approach. We take this approach one
step further by automating the procedure.

Ladan-Mozes \etal \cite{edya} present a protocol which are similar to the final
distributed protocol that we present. But they do not present any reasoning
about the correctness of the protocol, or an abstract automated procedure with
a set of invariants to generate further protocols.

Other techniques have been proposed which simplifies cache coherence using high
speed optical connections \cite{lipasti}. They revert back to atomic protocols
with a high-speed bus to lock addresses.
