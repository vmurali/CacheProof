\section{Related work}
\label{sec:related}

Model-checking based techniques have been used before in trying to prove cache
coherence protocols \cite{Clarke}, \cite{Chou}, \cite{ip}, \cite{McMillan},
\cite{pong}, \cite{Stern}. They try to tackle model-checking's state explosion
problem using innovative techniques, but the fundamental issue remains -- large
systems can not be verified by model-checking.

Zhang \etal proposed Fractal Coherence \cite{Zhang} for verifying protocols
using model-checking. A minimum system is verified and the protocol has to be
designed in such a way that if a subtree is replaced by a single node, it has
the same interface behavior (the fractal property). The difficulty in this
approach is that fractal property is not common in hierarchical protocols, extra
non-intuitive messages have to be passed around to ensure the property.

There has also been some work on using theorem proving techniques for proving
the correctness of directory-based cache coherence protocols \cite{loew},
\cite{park}. They rely on using abstraction functions to simplify the proof of
coherence protocols, which is similar to our approach. We take this approach one
step further by doing automating the procedure.

\cite{Beu} and\cite{edya} present protocols which is similar to the final
distributed protocol that we present. But they do not present any reasoning
about the correctness of the protocol, or an abstract automated procedure with
a set of invariants to generate further protocols.


