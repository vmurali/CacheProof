\section{Informal description of the proof}
\label{sec:informal-proof}

\newcommand{\msg}[5]{\text{$#1 \langle #2 \rightarrow #3, #4, #5 \rangle$}}
\newcommand{\Req}[4]{\msg{Req}{#1}{#2}{#3}{#4}}
\newcommand{\Resp}[4]{\msg{Resp}{#1}{#2}{#3}{#4}}
\newcommand{\Data}[4]{\msg{Data}{#1}{#2}{#3}{#4}}
\newcommand{\Mesg}[4]{\msg{Mesg}{#1}{#2}{#3}{#4}}
\newcommand{\FirstInv}{\emph{FirstInv}}
%\newcommand{\fourAngle}[5]{\text{$#1\langle#2,#3,#4,#5\rangle$}}
\newcommand{\ReqCore}[4]{\fourAngle{Req}{#1}{#2}{#3}{#4}}

\subsection{Atomic MSI protocol for 2-levels, no replacement evictions}

XXXXXX: This subsection is probably not going to go into the paper, since it's
too obvious. The definitions have been repeated in the later subsections.

We start with an atomic MSI protocol, with two levels of the memory hierarchy --
the L1 caches and the main memory. In this atomic protocol, the memory
controller can read and write the coherence states as well as the data of each
address present in the L1 caches. As usual, coherence state for an address in
the cache denotes the permissions that the cache has for the address: $M = \{R,
W\}, S = \{R\}, I = \{\}$.

Let us assume that the caches have enough capacity so that no line gets evicted
for the sake of replacement. Lines may however get evicted to enforce the
single-writer property, \ie only one L1 cache can modify a cache line at any
time. Figure~\ref{alg:msi-easy} shows the pseudo-code for the behaviors of the
memory controller and each of the caches to implement the MSI protocol, using
the notations given in Figure~\ref{fig:notation}. 

\begin{figure}
\begin{figure}[H]
\begin{boxedminipage}{\linewidth}
\begin{algorithmic}
\State Pick a request \ReqCore{c}{a}{x}{d} from core to cache $c$ \newline
$s.t.$ $state[c][a] < x$;
\ForAll {$c' \neq c,\; s.t.\; toCompatible(x) < state[c'][a]$}
  \If {$state[c'][a] = M$}
    \State $memory[a] \gets data[c'][a]$;
  \EndIf
  \State $state[c'][a] \gets toCompatible(x)$;
\EndFor
\If {$state[c][a] = I$}
  \State $data[c][a] \gets memory[a]$;
\EndIf
\State $state[c][a] \gets x$;
\end{algorithmic}
\end{boxedminipage}
\subcaption{Behavior of memory controller}
\label{alg:mem}
\end{figure}
\begin{figure}[H]
\begin{boxedminipage}{\linewidth}
\begin{algorithmic}
\State Pick a request \ReqCore{c}{a}{x}{d} from core to cache $c$ \newline
$s.t.$ $state[c][a] \not< x$;
\If {$state[c][a] = M$}
  \State $update(c, a, d)$;
\ElsIf {$state[c][a] = S$}
  \State $respond(c, a, d)$;
\EndIf
\State remove \ReqCore{c}{a}{x}{d}
\end{algorithmic}
\end{boxedminipage}
\subcaption{Behavior of cache $c$}
\end{figure}
\caption{Atomic 2-level MSI protocol: memory controller has instantaneous
access to all states and data of all L1 caches and data in the memory}
\label{alg:msi-easy}
\end{figure}

\begin{figure}
\begin{minipage}{\linewidth}
\begin{tabularx}{\linewidth}{|cX|}
\hline
$toCompatible(s)$ & If the state of one cache is $s$ for address $a$, the
highest state of another cache for the same $a$
$\newline
toCompatible(M) = I,\newline
toCompatible(S) = S,\newline
toCompatible(I) = M
$\\
\hline
$x < y$ & Does state $x$ have fewer permissions that state $y$?
$I < M, I < S, S < M$\\
\hline
$\ReqCore{c}{a}{x}{d}$ & The incoming request for cache $c$ from a core\\
\multicolumn{1}{|r}{$a$} & the address of the cache line\\
\multicolumn{1}{|r}{$x$} & the state of the cache line the request is expecting
(\ie the minimum permissions required)\\
\multicolumn{1}{|r}{$d$} & information about the word offset, the byte enables
and the data (only in the case of store)\\
\hline
$update(c, a, d)$ & Action denoting update of a cache line for address $a$ in
cache $c$ with the incoming request's information $d$\\
\hline
$respond(c, a, d)$ & Action denoting cache $c$ responding to a read request for
address $a$ with the incoming request's information $d$\\
\hline
\end{tabularx}
\subcaption{Abstraction functions}
\end{minipage}

\begin{minipage}{\linewidth}
\begin{tabularx}{\linewidth}{|cX|}
\hline
$state[c][a]$ & coherence state (from set $\{M, S, I\}$) of cache $c$ for
address $a$\\
$data[c][a]$ & data line of cache $c$ for address $a$\\
$memory[a]$ & data line of memory for address $a$\\
\hline
\end{tabularx}
\subcaption{Values stored by a cache for an address}
\end{minipage}
\caption{Notations used in Figure~\ref{alg:msi-easy}}
\label{fig:notation}
\end{figure}

XXXXXX: End of obvious subsection


Let us walk through our design methodology for our message-passing hierarchical
directory based protocols.

\subsection{Message-passing directory-based MSI protocol for 2-levels, no
replacement evictions}
For the sake of explanation, let us start with an MSI protocol, with just two
levels of the memory hierarchy -- the L1 caches and the main memory. We will
eventually build realistic protocols with various optimizations.

Let us assume that the caches have enough capacity so that no line gets evicted
for the sake of replacement. Lines may however get evicted to enforce the
single-writer property for cache coherence, \ie when one L1 cache has
permission to modify the data for an address, no other cache has read or write
permission for that address.
 
An L1 cache stores the information given in Figure \ref{table:lineinfo}. The
coherence state for an address denotes the permission that a cache has for the
address: $M = \{R, W\}, S = \{R\}, I = \{\}$. State $I$ is akin to the cache not
having the data corresponding to $a$, since it can neither read or write that
cache line.

\begin{figure}
\begin{tabularx}{\linewidth}{|cX|}
\hline
$state[c][a]$ & coherence state (from set $\{M, S, I\}$) of cache $c$ for
address $a$\\
$data[c][a]$ & data in cache $c$ for address $a$\\
\hline
\end{tabularx}
\caption{Values stored by a cache for an address}
\label{table:lineinfo}
\end{figure}

In order to serve a request from the processor, if a cache has to \emph{upgrade}
its state for address $a$, \ie change from $x$ to $y$ where $x > y$
(Figure \ref{table:fns}), it can do so only after ensuring that the new state is
\emph{compatible} with the states of other caches for that address, \ie other
caches are in a state less than or equal to $toCompatible(y)$
(Figure \ref{table:fns}). This is needed to ensure the single-writer property of
the MSI protocol -- if a cache's state is $M$ for an address $a$, every other cache
must be in $I$ for address $a$. But, the caveat is that the L1 cache can access
only its own state and data, as opposed to that of other caches.

\begin{figure}
\begin{tabularx}{\linewidth}{|cX|}
\hline
$toCompatible(s)$ & If the state of one cache is $s$ for address $a$, the
highest state of another cache for the same $a$
$\newline
toCompatible(M) = I,\newline
toCompatible(S) = S,\newline
toCompatible(I) = M
$\\
\hline
$x > y$ & Denotes if state $x$ has more permissions than state $y$.
$M > S, M > I, S > I$. Relations $\le, >, \ge$ are also defined appropriately\\
\hline
\end{tabularx}
\caption{Abstraction functions}
\label{table:fns}
\end{figure}

The memory controller has access to the data from the memory but not to the
state and data of the L1 caches. Instead, it only has access to a directory
which has a version of the coherence states of the caches for each address. The
version of the state in the directory may not be up to date, \ie may not be the
same as the one in the caches, all the time. Figure \ref{table:dirinfo} shows
the information stored in the memory and directory. We will use the word memory
and directory interchangeably for the 2-level protocol. The caches can not
communicate with each other but can communicate with the directory via messages.

\begin{figure}
\begin{tabularx}{\linewidth}{|cX|}
\hline
$dir[c][a]$ & directory's version of the coherence state of cache $c$ for address $a$\\
$memory[a]$ & data in the memory address $a$\\
\hline
\end{tabularx}
\caption{Values stored by a directory and memory for an address}
\label{table:dirinfo}
\end{figure}

Let's us assume that we have a system which guarantees the invariants given
in Figure \ref{FirstInv}.
\floatstyle{boxed}
\restylefloat{figure}

\begin{figure}
\begin{enumerate}
\item The cache can upgrade $state[c][a]$ from $x$ to $y$ where $y > x$, \ie
\emph{downgrade} its directory state only after it receives the appropriate
authorization from the directory for address $a$. \label{cwaitresp}
\item If the cache wants to upgrade $state[c][a]$ to $x$, there is a
message \Req{c}{Dir}{a}{x} which cache $c$ can send to the directory requesting
the upgrade authorization, and the cache will eventually be able to send
the request. \label{csendreq}
\item If the directory receives an upgrade request \Req{c}{Dir}{a}{x} from cache
$c$ for address $a$, then just before receiving the request, $dir[c][a] < x$.
\label{drecvreq}
\item The directory can change $dir[c][a]$ from $x$ to $y$ where $y < x$, \ie
\emph{downgrade} its directory state only after it receives the appropriate
authorization from cache $c$ for address $a$. \label{dwaitresp}
\item If the directory wants to downgrade $dir[c][a]$ to $x$, there is a
message \Req{Dir}{c}{a}{x} which the directory can send to cache $c$ requesting
the downgrade authorization, and the directory will eventually be able to send
the request. \label{dsendreq}
\item If the directory has sent a downgrade request for an address $a$ to cache
$c$, it will eventually get the downgrade authorization response for address $a$
from cache $c$ to downgrade to $x$, where $x$ is less than or equal to the state
to which the directory sent a request to downgrade to. \label{drecvresp}
\item If $dir[c][a] = M$ just before receiving a downgrade authorization from
cache $c$ for address $a$, then the cache would have sent the data along with
the authorization. \label{drecvdata}
\item If the directory wants to send an authorization message to upgrade the
state of address $a$ in cache $c$ to $x$ (\Resp{Dir}{c}{a}{x}), it will
eventually be able to send it. \label{dsendresp}
\item The directory can send the data from the memory along with the upgrade
authorization message if it wants to. \label{dsenddata}
\item The directory's version of the cache's state is always conservative, \ie
$\forall c, a,\; dir[c][a] \ge state[c][a]$. \label{conservative}
\end{enumerate}
\caption{Initial set of invariants \FirstInv}
\label{FirstInv}
\end{figure}

\floatstyle{plain}
\restylefloat{figure}

We will walk through how the MSI coherence protocol (with no replacement
evictions) can be implemented using the above invariants and argue why it is a
correct implementation.

The directory's role is to ensure the single-writer property and that the last
updated data is being read. In order to implement the cache coherence protocols
using the invariants of Figure \ref{FirstInv}, the directory's ensures that at
all times, if for some $c, dir[c][a] = M$, the for every other $c', dir[c'][a] =
I$.

If a cache gets a request from the core which can not be handled because of lack
of permissions, it (eventually) sends an upgrade request to the directory.
\FirstInv.\ref{csendreq} ensures that such a request can eventually be sent.

If the directory receives a request from cache $c$ to upgrade the state of
address $a$ to $x$, because of \FirstInv.\ref{drecvreq}, $dir[c][a] < x$ and
since $cache[c][a] \le dir[c][a]$ because of \FirstInv.\ref{conservative},
$cache[c][a] < x$ and hence the directory must send an upgrade authorization
response. The directory sends a request \Req{Dir}{c'}{a}{toCompatible(x)} to
every cache $c'$ for which $dir[c'][a] > toCompatible(x)$. Such a request can be
sent eventually because of \FirstInv.\ref{dsendreq}.

Because of \FirstInv.\ref{drecvresp}, the directory will eventually receive
responses from each of the requested caches $c'$
(\Resp{c'}{Dir}{a}{toCompatible(x)}) authorizing the downgrade of $dir[c'][a]$
to $toCompatible(x)$. If $dir[c'][a] = M$ just before receiving the
authorization, then by \FirstInv.\ref{drecvdata}, the directory would also
receive data from $c'$. Because the directory maintains the single-writer
invariant, every other cache $c''$ would have $dir[c''][a] = I$ just before
receiving the authorization. This implies for every other cache $c''$,
$state[c''][a] = I$ by the \FirstInv.\ref{conservative} invariant, and hence
$c'$ has the most up-to-date copy for address $a$. Thus the memory is updated
with the received up-to-date copy for address $a$.

Once all the authorizations have been received, for every cache $c'$,
$dir[c'][a] \le toCompatible(x)$ (other than the one that made the request $c$).
Since the \FirstInv.\ref{conservative} invariant ensures that $cache[c'][a] \le
dir[c'][a]$, we have $cache[c'][a] \le toCompatible(x)$, ensuring the
single-writer property (because $toCompatible(M) = I$). The directory can now
change $dir[c][a]$ to $x$ and send \Resp{Dir}{c}{a}{x} authorizing $c$ to
upgrade state of address $a$ to $x$ (\FirstInv.\ref{dsendresp} ensures that such
a message can be sent). Just before sending the authorization, if $dir[c][a] =
I$, it implies that $state[c][a] = I$ because of the
\FirstInv.\ref{conservative} invariant.  Thus the cache doesn't have data
corresponding to address $a$ and the directory has to supply the most up to date
version of the address (from memory, since the memory is up to date), which it
can because of \FirstInv.\ref{dsenddata}.

The \FirstInv\ invariants essentially makes it straight forward to design the
behavior of the directory . For the sake of argument, let
\FirstInv.\ref{drecvreq} not be guaranteed. Consider a scenario the cache $c$
which has sent a request \Req{c}{Dir}{a}{M} earlier but hasn't received the
upgrade authorization response yet. Meanwhile, $c$ sends another upgrade request
\Req{c}{Dir}{a}{S}. Let us assume that the directory processed the requests in
the order in which they were sent. Once the directory has finished processing
the first request, it changes its directory state $dir[c][a]$ to $M$. When it
receives the second request, it is not easy to decide if the directory has to
drop the request or to send another response authorizing the cache to change to
$S$.

The difficulty in message passing protocols comes from the fact that it is
difficult to guarantee the above invariants, even for a 2-level memory hierarchy
discussed above as shown in the examples below.

\subsubsection*{Example 1: Inability to send messages because of network
congestion:}
For instance, the ability of a cache or the directory to send messages
(\FirstInv.\ref{dsendreq}, \FirstInv.\ref{dsendresp}, \FirstInv.\ref{csendreq},
\etc) depends upon the availability of free buffers in the network; if the
network remains congested all the time, then it would not be possible to send the
messages, and the system would deadlock.

\subsubsection*{Example 2: Multiple upgrade requests from the cache for the same
address:}
If a cache is allowed to make multiple requests to upgrade the state of an
address (say to $M$ and then to $S$) before receiving the upgrade
authorizations, then it might violate \FirstInv.\ref{drecvreq} if the requests
were processed by the directory in the order they were received.

\subsubsection*{Example 3: On sending an upgrade request, the cache waits till
it gets an authorization:}
If the behavior of the caches were such that if a cache $c$ has sent an
upgrade request \Req{c}{Dir}{a}{x} to the directory, it keeps waiting for the
upgrade authorization response \Resp{Dir}{c}{a}{x} before processing any
requests from the directory for address $a$, then the cache coupled with our
simple directory implementation would deadlock. Consider a scenario with two
caches $X$ and $Y$, with both the caches in state $S$ for address $a$. Both the
caches got a store request from the processor and want to upgrade their states
to $M$, and hence send \Req{X}{Dir}{a}{M} and \Req{Y}{Dir}{a}{M}, respectively,
to the directory. Let us say the directory receives \Req{Y}{Dir}{a}{M} first,
and in order to process it, according to our implementation of the directory,
sends a downgrade request \Req{Dir}{X}{a}{I} to cache $X$. But the cache 
does not process any request from the directory for address $a$, according to
our assumption, and hence will never send the downgrade authorization
\Resp{X}{Dir}{a}{I} to the directory, violating \FirstInv.\ref{drecvresp}, and
hence causing a deadlock.

\subsubsection*{Example 4: At the directory, authorization responses are blocked
by upgrade requests:}
Consider a scenario where two caches $X$ and $Y$ are both in state $S$ for address
$a$. $Y$ got a store request from the processor and sends an upgrade request
\Req{Y}{Dir}{a}{M} to the directory. The directory receives that request, and
sends a downgrade request \Req{X}{Dir}{a}{I} to $X$. $X$ sends a downgrade
authorization \Resp{X}{Dir}{a}{I} to the directory on receiving the downgrade
request. But the directory can not receive the authorization because it is
blocked by other requests (including \Req{Y}{Dir}{a}{M}). This violates
\FirstInv.\ref{drecvresp} and hence creates a deadlock.

These examples show that if the \FirstInv\ invariants are violated, then it
either breaks cache coherence (by creating deadlocks) or it makes correctly
designing the directory very difficult.

Before we show how the \FirstInv\ invariants can be guaranteed (which requires
the description of the behavior of the L1 caches), let us consider a
slightly more complicated system where caches can evict lines for replacement.

\subsection{Message-passing directory-based MSI protocol for 2-levels, with
replacement evictions}
We again have just two levels of memory hierarchy -- the L1 caches and the main
memory. But the caches have finite capacity. If receives a request from the core
for an address $a$ and has no space to accommodate address $a$, it evicts
another address $a'$ to make room for $a$.

In the previous scenario, where the caches didn't have to evict for replacement,
a cache only had to send downgrade authorizations to the directory on receiving
a corresponding request. But in the new scenario, a cache can send a downgrade
notification to the directory on evicting an address. Such a notification is
useful in keeping the directory's version of the cache's state up to date, so we
will restrict the family of directory-based protocols that we discuss to those
where the caches send such a notification, voluntarily.

The directory handles this notification in the exact manner by which it handled
a downgrade authorization response from the cache, so the notification message
is the same as a downgrade authorization message from cache $c$ for authorizing
the downgrade of the $dir[c][a]$ to $I$ (\Resp{c}{Dir}{a}{I}).

We will now discuss a behavior of the L1 cache that would violate the
invariant that the directory's version of the states of the caches for each
address is conservative.

\subsubsection*{Example 5: Cache sends voluntary notifications
(\Resp{c}{Dir}{a}{I}) indiscriminately}
Consider a system with two caches $X$ and $Y$. Let $state[X][a] = S, state[Y][a]
= I$ and let $dir[X][a] = S, dir[Y][a] = I$. $X$ receives a store request from
the core, and hence sends an upgrade request \Req{X}{Dir}{a}{M} to the
directory. Say $X$ was a non-blocking cache, and hence receives another load
request for address $a'$ from the core before receiving an upgrade authorization
for address $a$. $X$ does not have space in the cache for $a'$, and it decides
to evict address $a$, and sends \Resp{X}{Dir}{a}{I} to the directory, changing
$state[c][a]$ to $I$.  Meanwhile, the directory receives the request
\Req{X}{Dir}{a}{M} and since $dir[Y][a] \le I$, it responds to $X$ with an
upgrade authorization \Resp{Dir}{X}{a}{M} (along with the data). Later, the
directory receives the notification \Resp{X}{Dir}{a}{I} from $X$, and downgrades
$dir[X][a]$ to $I$. Meanwhile, $X$ receives the upgrade authorization
\Resp{Dir}{X}{a}{M} and changes $state[X][a]$ to $M$. Now,
\FirstInv.\ref{conservative} is broken, since $state[X][a] = M$ and $dir[X][a] =
I$. Let $Y$ get a load request from its core, and hence sends a
\Req{Y}{Dir}{a}{S} to the directory. The directory on receiving the request,
sends an upgrade authorization to $Y$ because $dir[X][a] = I$. This violates the
single-writer property and hence coherence.
