\section{Caches as a distributed system of multi-threaded execution engines}
\label{sec:threads}

The memory hierarchy in a processor can be specified as a collection of nodes
(Figure \ref{distNodes}). Each node has local state associated with it. The
state may contain the cache and the directory among other book-keeping
information. A node also has multiple input and multiple output channels
through which it can send and receive messages. Most of the channels are
connected to a network, which delivers messages between nodes
%; the message delivery order is left unspecified
. Some of the channels may not be connected to the network; they represent
inputs and outputs to the memory system, \viz the requests from and responses to
cores.

The behavior of each node can be best described as a system of suspendable local
threads, with a scheduler local to that node.  A thread can be in one of the two
states: \emph{executing} or \emph{suspended}. Once a thread finishes executing,
the scheduler does one of the following: a) picks one of suspended threads which
has become ready to execute due to changed conditions or b) creates a new thread
to handle an incoming message. An executing thread can read the local state
of the node, send messages to output channels, receive messages from input
channels or spawn more threads. The input and output channels are blocking, so
if a thread tries to send to a full output channel or receive from an empty
input channel, the thread would not be able to proceed. The thread can also
fail to proceed due to some condition in the local state of the node. In
all these cases, the thread gets suspended. It remains suspended till the
required conditions are met (for instance, the output channel becomes free, the
input channel receives a pertinent message, or other local state changes),
Once the required conditions for a suspended thread are met the scheduler may
pick it for execution. The execution is resumed from the exact point where it
was suspended. Once the thread finishes its execution, all the book-keeping
resources associated with that thread is freed up and reclaimed.

Incoming messages remain in the channel until they are explicitly removed by
the thread dealing with that message. If the scheduler sees an incoming message
that requires a new thread to be created for handling, and if the required
book-keeping resources are available, then a thread is created.
Different threads require different amounts of book-keeping resources. If the
required book-keeping resources are not available, the scheduler might pick
another suspended thread which is ready to execute, or if that's not possible,
spin-waits.  The threads and incoming messages have a memory address associated
with them. For an incoming message that does not require a new thread to be
created, its address is used by the scheduler to decide if a suspended thread
is ready to execute. Similarly, the address associated with a thread that has
just finished executing is used to determine if the local state of the node
associated with that address has changed so that a suspended thread associated
with that address can resume execution.  

Note that the thread analogy we described above has nothing to do with software
threads. We are describing the behavior of a cache controller using the familiar
notion of suspendable threads. The threads essentially correspond to the various
requests being handled by a cache controller's Miss Status Handler Register
(MSHR).

Earlier, each thread is started,, because of an incoming message, with a
handler \textbf{procedure} particular to the type of the message. A procedure
groups together actions which include state updates ($s \gets value$),
\textbf{send} messages into output channels and \textbf{pop} messages from
input channels. \textbf{send} and \textbf{pop} are blocking calls, as mentioned
earlier. Procedures can \textbf{call} other procedures but they can not be
recursive.
