\section{Initialize}
\label{sec:init}

Cache coherence protocols maintain coherence by ensuring the following two
invariants:

\begin{inv}
\textbf{Single-writer Invariant}: When a cache has permission to modify data for
an address, no other cache has read or write permission for that
address.
\label{singleWriter}
\end{inv}

\begin{inv}
\textbf{Read-from-last-writer}: When a cache reads data for an address, it reads
the last modified version of the data for the address. If the data for an
address has not been modified, the cache reads the data from memory.
\label{lastRead}
\end{inv}

Because of the requirement to maintain the Single-writer property (Invariant
\ref{singleWriter}), it makes sense to talk about compatibility of the states of
two caches for an address. Two coherence states $x$ and $x$ are \emph{compatible}
if two caches having the same address in states $x$ and $x$ do not violate the
single-writer property.

\section{MSI protocol}
\label{sec:msi}

\newcommand{\printall}[5]{\text{$#1 \langle #2 \rightarrow #3, #4, #5 \rangle$}}
\newcommand{\Req}[4]{\printall{Req}{#1}{#2}{#3}{#4}}
\newcommand{\Resp}[4]{\printall{Resp}{#1}{#2}{#3}{#4}}
\newcommand{\Data}[5]{\printall{Data}{#1}{#2}{#3}{#4}{#5}}

\newcommand{\lett}{\textbf{let}}
\newcommand{\send}{\textbf{send}}
\newcommand{\receive}{\textbf{receive}}
\newcommand{\remove}{\textbf{remove}}
\newcommand{\assert}{\textbf{assert}}
\newcommand{\call}{\textbf{call}}

Let us walk through our framework using the example of an MSI protocol. We will
first discuss a system with two levels of memory hierarchy -- private L1 caches
for each core and a shared memory. 

\begin{figure}
\begin{tabularx}{\linewidth}{|cX|}
\hline
$c.state[a]$ & coherence state (from set $\{M, S, I\}$) of cache $c$ for address $a$\\
$c.data[a]$ & data in cache $c$ for address $a$\\
$memory[a]$ & data in memory for address $a$\\
\hline
\end{tabularx}
\caption{Notations}
\label{table:lineinfo}
\end{figure}

\begin{wrapfigure}{r}{.4\linewidth}
\centering
\begin{tabular}{|c|ccc|}
\hline
& $M$ & $S$ & $I$\\
\hline
$M$ & $=$ & $>$ & $>$\\
$S$ & $<$ & $=$ & $>$\\
$I$ & $<$ & $<$ & $=$\\
\hline
\end{tabular}
\caption{$<$ and other relations for MSI states}
\label{msi<}
\end{wrapfigure}

Each L1 cache stores the information given in Figure \ref{table:lineinfo}. The
coherence state for an address denotes the (usual) permission that a cache has
for the address: $M = \{R, W\}, S = \{R\}, I = \{\}$. We can define a $<$
relation for coherence state values based on the permissions the states
represent. This is shown in Figure \ref{msi<}. State $I$ is akin to address $a$
not present in the cache.

\begin{wrapfigure}{r}{.46\linewidth}
\begin{subfigure}{.27\linewidth}
\centering
\begin{tabular}{|ccc|}
\hline
$M$ & $\rightarrow$ & $I$\\
$S$ & $\rightarrow$ & $S$\\
$I$ & $\rightarrow$ & $I$\\
\hline
\end{tabular}
\caption{$toCompatible$}
\label{toCompat}
\end{subfigure}
~~~~~~~~~~~~~~
\begin{subfigure}{.21\linewidth}
\centering
\begin{tabular}{|c|c|}
\hline
$M$ & $\checkmark$\\
$S$ & $\times$\\
$I$ & $\times$\\
\hline
\end{tabular}
\caption{$isModified$}
\label{isModified}
\end{subfigure}
\caption{$toCompatible$ and $isModified$ mappings}
\label{fig:msimap}
\end{wrapfigure}

If a cache $c$ receives a request for an address $a$ from the processor, it can
service the request straightaway if it has enough permissions for address $a$.
Otherwise, it has to \emph{upgrade} to a state $x$, which has enough
permissions. The cache can not upgrade its state arbitrarily, it has to ensure
that such a state change does not violate the single-writer property (Invariant
\label{singleWriter}), by \emph{downgrading}, \ie lowering the states of other
caches appropriately. We define a $toCompatible(x)$ mapping (Figure
~\ref{toCompat}), which specifies the highest state a cache can be, if another
cache is in state $x$ for the same address. We will also define a
$isModified(x)$ mapping to specify if an address in state $x$ in a cache can
have data different from what's present in the memory.

\floatstyle{boxed}
\restylefloat{figure}

\begin{figure}

\begin{subfigure}{\linewidth}
\begin{boxedminipage}{\linewidth}
\begin{algorithmic}
\Proc {atomicDowngrade}{$c, a, x$}
  \If {$c.state[a] > x$}
    \If {$isModified(c.state[a])$}
      \State $memory[a] \gets c.data[a]$;
    \EndIf
    \State $c.state[a] \gets x$;
  \EndIf
\EndProc
\end{algorithmic}
\end{boxedminipage}
\caption{Downgrade address $a$ in cache $c$ to $x$}
\label{atomicDowngrade}
\end{subfigure}

\begin{subfigure}{\linewidth}
\begin{boxedminipage}{\linewidth}
\begin{algorithmic}
\Proc {atomicDowngradeMany}{$cs, a, x$}
  \ForAll {$c \in cs,\; s.t.,\; c.state[a] > x$}
    \If {$isModified(c.state[a])$}
      \State $memory[a] \gets c.data[a]$;
    \EndIf
    \State $c.state[a] \gets x$;
  \EndFor
\EndProc
\end{algorithmic}
\end{boxedminipage}
\caption{Downgrade address $a$ in caches $cs$ to $x$}
\label{atomicManyDowngrade}
\end{subfigure}

\begin{subfigure}{\linewidth}
\begin{boxedminipage}{\linewidth}
\begin{algorithmic}
\Proc {atomicUpgrade}{$c, a, x$}
  \If {$c.state[a] < x$}
    \If {$c.state[a] = I$}
      \State $c.data[a] \gets memory[a]$;
    \EndIf
    \State $c.state[a] \gets x$;
  \EndIf
\EndProc
\end{algorithmic}
\end{boxedminipage}
\caption{Upgrade address $a$ in cache $c$ to $x$}
\label{atomicUpgrade}
\end{subfigure}

\begin{subfigure}{\linewidth}
\begin{boxedminipage}{\linewidth}
\begin{algorithmic}
\If {$c$ wants to upgrade state of address $a$ to $x$}
  \If {$a$ is not present in $c$}
    \State Choose address $a'$ for eviction;
    \State \call{} {\textsc{atomicDowngrade}($c, a', I$)};
    \State Replace $a'$ with $a$ (initializing $c.state[a]$ to $I$);
  \EndIf
  \State \call{} \parbox[t]{\dimexpr\linewidth-\algorithmicindent}
	{\textsc{atomicDowngradeMany}
            ($\newline allCaches \smallsetminus \{c\}, a, $ $toCompatible(x)$)};
\State \call{} \textsc{atomicUpgrade}($c, a, x$);
\EndIf
\end{algorithmic}
\end{boxedminipage}
\caption{Putting it all together}
\label{atomicPutting}
\end{subfigure}

\caption{Atomic 2-level MSI protocol}
\label{atomic}
\end{figure}

\textsc{atomicDowngrade2} (Figure \ref{atomicDowngrade}),
\textsc{atomicDowngradeMany2} (Figure \ref{atomicDowngradeMany}) and
\textsc{atomicUpgrade2} (Figure \ref{atomicUpgrade}) specify the exact
procedure for atomically downgrading the state of exactly one cache, atomically
downgrading the states of several caches, and atomically upgrading the state of
one cache, respectively. Figure \ref{atomic}, specifies the complete protocol
making use of these procedures. If two or more caches wish to upgrade their
states, they get serialized -- only one cache can perform the actions in Figure
\ref{atomic} at a time. Bus-based protocols essentially behave in this manner,
by locking the bus till a cache completes the update procedure.  The purpose of
presenting Figures \ref{atomicDowngrade}, \ref{atomicUpgrade} and \ref{atomic}
is to familiarize the reader with our notation. 

In a directory setting, the cache and have direct access only to their local
states and local data. The directory maintains its version of the states of each
cache for each address, denoted by $dir[c][a]$ for cache $c$, address $a$. This
is the only state the directory can access directly in addition to the memory.
We would be using the terms directory and memory interchangeably in this
section. In order for a cache to change states of other caches, or access their
data, \etc, messages have to be passed between the caches. Let us assume that
only communication between the directory and a cache is allowed, caches can not
communicate with each other without going through the directory.

If a cache $c$ wants to upgrade to from state $x$ state $x$ for address $a$, it
can send a request \Req{c}{Dir}{a}{x} to the directory asking for permissions
to upgrade its state to $x$. It can get a response \Resp{Dir}{c}{a}{x} giving
it permissions to upgrade its state for address $a$ to $x$.

A directory can send a request \Req{Dir}{c}{a}{x} to cache $c$ asking it to
downgrade the state for address $a$ to $x$. A directory can get a response from
cache $c$ \Resp{c}{Dir}{a}{x}, telling the directory that the $c$ has
downgraded the state for address $a$ to $x$.

Data is also passed between the caches and the memory: \Data{Dir}{c}{a}{d} for
data from the memory to the cache, and \Data{c}{Dir}{a}{d} for data from the
cache to the memory.

\begin{figure}

\begin{subfigure}{\linewidth}
\begin{boxedminipage}{\linewidth}
\begin{algorithmic}
\If {$c$ wants to update the state of address $a$ to $x$}
  \If {$a$ is not present in $c$}
    \State Choose address $a'$ for eviction;
    \State \textsc{downgrade}($c, a, I$);
    \State Replace $a'$ with $a$ (initializing $c.state[a]$ to $I$);
  \EndIf
  \State \call{} \textsc{upgrade}($c, Dir, a, x$);
\EndIf
\end{algorithmic}
\end{boxedminipage}
\caption{Cache $c$ upgrading state of address $a$ to $x$}
\label{cacheUp}
\end{subfigure}

\begin{subfigure}{\linewidth}
\begin{boxedminipage}{\linewidth}
\begin{algorithmic}
\If {$Dir$ receives \Req{c}{Dir}{a}{x}}
  \State \call{} \parbox[t]{\dimexpr\linewidth-\algorithmicindent}
           {\textsc{downgradeChildren}($\newline Dir, allCaches
              \smallsetminus \{c\}, a, toCompatible(x)$);}
  \If {$dir[c][a] = I$}
    \State \send{} \Data{Dir}{c}{a}{memory[a]};
  \EndIf
  \State \send{} \Resp{Dir}{c}{a}{x};
  \State $dir[c][a] \gets x$;
  \State \remove{} triggering request \Req{c}{Dir}{a}{x};
\EndIf
\end{algorithmic}
\end{boxedminipage}
\caption{Directory receiving an upgrade to $x$ request for address $a$}
\label{dirDown}
\end{subfigure}

\begin{subfigure}{\linewidth}
\begin{boxedminipage}{\linewidth}
\begin{algorithmic}
\If {$c$ receives \Req{Dir}{c}{a}{x}}
  \State \call{} \textsc{downgrade}($c, a, x$);
  \State \remove{} triggering request \Req{Dir}{c}{a}{x};
\EndIf
\end{algorithmic}
\end{boxedminipage}
\caption{Cache receiving a downgrade to $x$ request for address $a$}
\label{cacheDown}
\end{subfigure}

\begin{subfigure}{\linewidth}
\begin{boxedminipage}{\linewidth}
\begin{algorithmic}
\If {$Dir$ receives \Resp{c}{Dir}{a}{x}}
  \If {$isModified(dir[c][a])$}
    \State \receive{} and \remove{} \Data{c}{Dir}{a}{d};
    \State $memory[a] \gets d$;
  \EndIf
  \State $dir[c][a] \gets x$;
\EndIf
\end{algorithmic}
\end{boxedminipage}
\caption{Directory receiving a downgrade to $x$ response for address $a$
without sending a request}
\label{dirDown}
\end{subfigure}

\caption{2-level MSI protocol in the directory setting}
\label{msidir}
\end{figure}

\begin{figure}

\begin{subfigure}{\linewidth}
\begin{boxedminipage}{\linewidth}
\begin{algorithmic}
\Proc {downgrade}{$c, a, x$}
  \If {$c.state[a] > x$}
    \State \send{} \Resp{c}{c.parent}{a}{x};
    \If {$isModified(c.state[a])$}
      \State \send{} \Data{c}{c.parent}{a}{c.data[a]};
    \EndIf
    \State $c.state[a] \gets x$;
  \EndIf
\EndProc
\end{algorithmic}
\end{boxedminipage}
\caption{Downgrade address $a$ in cache $c$ to $x$}
\label{downgrade}
\end{subfigure}

\begin{subfigure}{\linewidth}
\begin{boxedminipage}{\linewidth}
\begin{algorithmic}
\Proc {downgradeChildren}{$p, cs, a, x$}
  \ForAll {$c \in cs,\; s.t.,\; p.dir[c][a] > x$}
    \State \send{} \Req{p}{c}{a}{x};
    \State \receive{} \Resp{c}{p}{a}{z};
    \If {$isModified(p.dir[c][a])$}
      \State \receive{} \Data{c}{p}{a}{d};
      \State $p.data[a] \gets d$;
    \EndIf
    \State $p.dir[c][a] \gets z$;
  \EndFor
\EndProc
\end{algorithmic}
\end{boxedminipage}
\caption{Parent $p$ requesting a subset of its children $cs$ to downgrade state
of address $a$ to $x$}
\label{downgrademany}
\end{subfigure}

\begin{subfigure}{\linewidth}
\begin{boxedminipage}{\linewidth}
\begin{algorithmic}
\Proc {upgrade}{$c, p, a, x$}
  \If {$c.state[a] < x$}
    \State \send{} \Req{c}{p}{a}{x};
    \State \receive{} \Resp{p}{c}{a}{z};
    \If {$c.state[a] = I$}
      \State \receive{} \Data{p}{c}{a}{d};
      \State $c.data[a] \gets d$;
    \EndIf
    \State $c.state[a] \gets z$;
  \EndIf
\EndProc
\end{algorithmic}
\end{boxedminipage}
\caption{Child $c$ requesting its parent $p$ permissions to upgrade to $x$ for
address $a$}
\label{upgrade}
\end{subfigure}

\caption{Helper functions for hierarchical MSI protocol}
\label{multihelper}
\end{figure}

\begin{figure}

\begin{subfigure}{\linewidth}
\begin{boxedminipage}{\linewidth}
\begin{algorithmic}
\If {$p$ receives \Req{c}{p}{a}{x}}
  \If {$a$ is not present in $p$}
    \State Choose address $a'$ for eviction;
    \State \lett{} $oldState = p.state[a']$;
    \State \call{} \parbox[t]{\dimexpr\linewidth-\algorithmicindent}{
             \textsc{downgradeChildren}($\newline p, p.children, a', I$);}
    \State \call{} \textsc{downgrade}($p, a', I$);
    \State Replace $a'$ with $a$, initializing $state[c][a]$ to $I$;
  \EndIf
  \State \call{} \textsc{upgrade}($p, a, x$);
  \State \call{} \parbox[t]{\dimexpr\linewidth-\algorithmicindent}
            {\textsc{downgradeChildren}($\newline p,
                p.children \smallsetminus \{c\}, a, toCompatible(x)$)}
  \State \send{} \Resp{p}{c}{a}{x};
  \State \remove{} triggering request \Req{c}{p}{a}{x};
\EndIf
\end{algorithmic}
\end{boxedminipage}
\caption{Cache $p$ receiving an upgrade to $x$ request from its child $c$ for
address $a$}
\label{dirreq}
\end{subfigure}

\begin{subfigure}{\linewidth}
\begin{boxedminipage}{\linewidth}
\begin{algorithmic}
\If {$c$ receives \Req{c.parent}{c}{a}{x}}
  \State \textsc{downgradeChildren}($c, c.children, a, x$);
  \State \textsc{downgrade}($c, a, x$);
  \State \remove{} triggering request \Req{c.parent}{c}{a}{x};
\EndIf
\end{algorithmic}
\end{boxedminipage}
\caption{Cache $c$ receiving a downgrade to $x$ request from its parent for address $a$}
\label{dirreq}
\end{subfigure}

\begin{subfigure}{\linewidth}
\begin{boxedminipage}{\linewidth}
\begin{algorithmic}
\If {$p$ receives \Resp{c}{p}{a}{x}}
  \If {$isModified(p.dir[c][a])$}
    \State \receive{} \Data{c}{p}{a}{d};
    \State $p.data[a] \gets d$;
  \EndIf
  \State \remove{} response \Resp{c}{p}{a}{x};
\EndIf
\end{algorithmic}
\end{boxedminipage}
\caption{Cache $p$ receiving a downgrade to $x$ response from its child $c$ for
address $a$ without sending a request}
\label{dirreq}
\end{subfigure}

\caption{Hierarchical MSI protocol}
\label{multimsi}
\end{figure}

%\begin{figure}
%
%\begin{subfigure}{\linewidth}
%\begin{boxedminipage}{\linewidth}
%\begin{algorithmic}
%\Proc {downgrade2}{$c, a, x$}
%  \If {$c.state[a] > x$}
%    \State \send{} \Resp{c}{Dir}{a}{x};
%    \If {$isModified(c.state[a])$}
%      \State \send{} \Data{c}{Dir}{a}{c.data[a]};
%    \EndIf
%    \State $c.state[a] \gets x$;
%  \EndIf
%\EndProc
%\end{algorithmic}
%\end{boxedminipage}
%\caption{Cache $c$ downgrading to $x$}
%\label{downgrade2}
%\end{subfigure}
%
%\begin{subfigure}{\linewidth}
%\begin{boxedminipage}{\linewidth}
%\begin{algorithmic}
%\Proc {downgradeMany2}{$cs, a, x$}
%  \ForAll {$c \in cs,\; s.t.\; dir[c][a] > x$}
%    \State \send{} \Req{Dir}{c}{a}{x};
%    \State \receive{} \Resp{c}{Dir}{a}{z};
%    \State // Should $z = x$?
%    \If {$isModified(dir[c][a])$}
%      \State \receive{} \Data{c}{Dir}{a}{d};
%      \State $memory[a] \gets d$;
%    \EndIf
%    \State $dir[c][a] \gets z$;
%  \EndFor
%\EndProc
%\end{algorithmic}
%\end{boxedminipage}
%\caption{$Dir$ requesting caches $cs$ to downgrade to $x$}
%\label{downgrade2}
%\end{subfigure}
%
%\begin{subfigure}{\linewidth}
%\begin{boxedminipage}{\linewidth}
%\begin{algorithmic}
%\Proc {upgrade2}{$c, a, x$}
%  \If {$c.state[a] < x$}
%    \State \send{} \Req{c}{Dir}{a}{x};
%    \State \receive{} \Resp{Dir}{c}{a}{z};
%    \State // Should $z = x$?
%    \If {$dir[c][a] = I$}
%      \State \send{} \Data{Dir}{c}{a}{memory[a]};
%    \EndIf
%    \State $dir[c][a] \gets z$;
%  \EndIf
%\EndProc
%\end{algorithmic}
%\end{boxedminipage}
%\caption{Cache $c$ requesting $Dir$ to upgrade to $x$}
%\label{upgrade2}
%\end{subfigure}
%
%\caption{Helper functions for 2-level MSI protocol in a directory setting}
%\label{helper2}
%\end{figure}

%%Figures \ref{fig:msi-cache-wrong}, \ref{fig:msi-dir-wrong},
%%\ref{fig:msi-cache2-wrong} and \ref{fig:msi-dir2-wrong} gives a \naive{}
%%translation of the atomic MSI protocol of Figure \ref{fig:msi-atomic} for the
%%directory setting.
%We will now give procedures that each of the caches (and the directory) should
%follow in the directory setting.  We need at least two procedures, one for a
%cache to initiate an upgrade request (Figure \ref{alg:msi-cache-wrong}, and one
%for the directory to handle an upgrade request from a cache (Figure
%\ref{alg:msi-dir-wrong}). Since the directory sends downgrade requests to
%caches, we need a procedure for a cache to handle a downgrade request (Figure
%\ref{alg:msi-cache2-wrong}). Finally, a cache can evict an address and replace
%it with another address as seen in the atomic protocol. This involves the cache
%sending a downgrade notification to the directory and this should be handled by
%the directory (Figure \ref{alg:msi-dir2-wrong}). In our framework, we use the
%same message type for both downgrade notification and downgrade response -- the
%choice is not arbitrary; we will see later that this reduces the time needed to
%wait for downgrade response messages, and hence improves overall performance.
%\send{}, \receive{}, and \remove{} are blocking. Let's also assume that there is
%a point-to-point ordering for each type of message, but not between different
%types of messages. For example, requests from caches to directories have FIFO
%ordering, and so do responses from caches to directories, but between requests
%and responses, there is no point to point ordering.
%
%These procedures are written assuming that the directory notion of a cache's
%state is identical with the actual state of the cache. We use \assert{}s to
%denote what the procedure expects. For instance, in Figure
%\ref{alg:msi-dir-wrong}, it expects a response from cache $c'$ for address $a$
%saying that the cache has downgraded its state from $dir[c'][a]$ to
%$toCompatible(x)$. Similarly, in Figure \ref{alg:msi-cache2-wrong}, it expects a
%request from the directory to downgrade from its current state, not from a
%different state. Most of these assumptions are wrong, as we show in the
%following.
%
%Consider the following scenario, with the assumption that the directory's state
%and the cache's state match exactly. Let cache $c$ decide to upgrade its state
%from $S$ to $M$. It sends an upgrade request to the directory and waits for a
%response. Meanwhile, another cache $c'$ also decides to upgrade its state from
%$S$ to $M$, sending its own upgrade request and waiting for a response. The
%directory received the request from $c$ first and proceeds to service it
%according to Figure \ref{alg:msi-dir-wrong}. It sends a downgrade request to
%cache $c'$ to downgrade its state from $S$ to $toCompatible(M)$, \ie $I$. But
%$c'$ is ``busy'' waiting for a downgrade response from the directory. If cache
%$c'$ doesn't respond, the directory can not start processing the upgrade request
%from $c'$ leading to a deadlock.
%
%The reason for bringing out this scenario first was to introduce our approach to
%reason about coherence protocols. Each of the procedures in Figures
%\ref{alg:msi-cache-wrong} to \ref{alg:msi-cache2-wrong} defines what we call as
%``transactions''. An incoming message (request or response) or in the case of
%the L1 caches, the decision to upgrade its state (which in turn is triggered by
%a request from the core) spawns a new transaction that executes the action
%corresponding to the handling of that message. It the transaction starts waiting
%(say because the transaction could not send a message as the corresponding
%buffer was full, or it could not receive a message as the corresponding message
%has not arrived), the transaction ``yields control''; if the waiting condition
%has been fulfilled (say the required response arrived, or the sending buffer has
%space), then the transaction will eventually be resumed from where it stopped
%for starting to wait. In terms of hardware implementation, this can easily be
%implemented using MSHRs -- the MSHR stores the position in the procedure from
%where the transaction has to resume. XXX Expand, be more clear. say mshrs can be
%used for needed for waiting transactions only. it can be stored at a per
%directory level too.
%
%\begin{figure}
%\begin{algorithmic}
%\Proc {downgrade}{$n, cs, a, x$}
%  \ForAll {$c \in cs,\; s.t.\; n.dir[c][a] > x$}
%    \State \send{} \Req{n}{c}{a}{x};
%    \While {$z > x$}
%      \State \receive{} \Resp{c}{n}{a}{z};
%      \If {$isModified(n.dir[c][a])$}
%        \State \receive{} \Data{c}{n}{a}{d};
%        \State $n.data[a] \gets d$;
%      \EndIf
%      \State $n.dir[c][a] \gets z$;
%    \EndWhile
%  \EndFor
%\EndProc
%\end{algorithmic}
%\caption{Downgrading the states of the $cs$ subset of the children of a cache to $x$, for address $a$}
%\label{downgradefull}
%\end{figure}
%
%\begin{figure}
%\begin{algorithmic}
%\Proc {upgrade}{$n, a, x$}
%  \If {$n.state[a] < x$}
%    \State \send{} \Req{n}{P}{a}{x};
%    \While {$z < x$}
%      \State \receive{} \Resp{P}{n}{a}{z};
%      \If {$n.state[c][a] = I$}
%        \State \receive{} \Data{P}{n}{a}{d};
%        \State $n.data[a] \gets d$;
%      \EndIf
%      \State $n.state[c][a] \gets z$;
%    \EndWhile
%  \EndIf
%\EndProc
%\end{algorithmic}
%\caption{Upgrading the state of a cache to $x$ for address $a$, by requesting its parent}
%\label{upgradefull}
%\end{figure}
%
%\begin{figure}
%\begin{algorithmic}
%\If {$n$ receives \Req{c}{n}{a}{x}}
%  \If {$a$ is not present in $n$}
%    \State Choose address $a'$ for eviction;
%    \State \lett{} $oldState = n.state[a']$;
%    \State \call{} \textsc{downgrade}($n, n.children, a', I$);
%    \State \send{} \Resp{n}{P}{a'}{I};
%    \If {$isModified(oldState[a'])$}
%      \State \send{} \Data{n}{P}{a'}{n.data[a']};
%    \EndIf
%    \State \send{} \Resp{n}{P}{a'}{I};
%    \State Replace $a'$ with $a$, initializing $state[c][a]$ to $I$;
%  \EndIf
%  \State \call{} \textsc{upgrade}($n, a, x$);
%  \State \lett{} $allButC = n.children \setminus \{c\}$;
%  \State \call{} \textsc{downgrade}($n, allButC, a, toCompatible(x)$);
%  \State \send{} \Resp{n}{c}{a}{x};
%  \State \remove{} triggering request \Req{c}{n}{a}{x};
%\EndIf
%\end{algorithmic}
%\caption{A cache handling a request from one of its children}
%\label{dirreq}
%\end{figure}
%
%\begin{figure}
%\begin{algorithmic}
%\If {$n$ receives \Req{P}{n}{a}{x}}
%  \State \lett{} $oldState = n.state[a']$;
%  \State \call{} \textsc{downgrade}($n, n.children, a, x$);
%  \If {$isModified(oldState[a])$}
%    \State \send{} \Data{n}{P}{a}{n.data[a]};
%  \EndIf
%  \State \remove{} triggering request \Req{P}{n}{a}{x};
%\EndIf
%\end{algorithmic}
%\caption{A cache handling a request from its parent}
%\label{dirreq}
%\end{figure}
%
%\begin{figure}
%\begin{algorithmic}
%\If {$n$ receives \Resp{c}{n}{a}{x}}
%  \If {$isModified(n.dir[c][a])$}
%    \State \receive{} \Data{c}{n}{a}{d};
%    \State $n.data[a] \gets d$;
%  \EndIf
%  \State \remove{} response \Resp{c}{n}{a}{x};
%\EndIf
%\end{algorithmic}
%\caption{A cache handling a non-waiting response from one of its children}
%\label{dirreq}
%\end{figure}

%%In the above scenario, the transaction which sent the request to 
%
%\newcommand{\fourAngle}[6]{\text{$#1\langle#2\rightarrow#3,#4,#5\rightarrow#6\rangle$}}
%\newcommand{\threeAngle}[5]{\text{$#1\langle#2\rightarrow#3,#4,#5\rangle$}}
%
%%\newcommand{\Req}[5]{\fourAngle{Req}{#1}{#2}{#3}{#4}{#5}}
%\newcommand{\Req}[4]{\threeAngle{Req}{#1}{#2}{#3}{#4}}
%\newcommand{\Resp}[4]{\threeAngle{Resp}{#1}{#2}{#3}{#4}}
%
%\newcommand{\msg}[5]{\text{$#1 \langle #2 \rightarrow #3, #4, #5 \rangle$}}
%%\newcommand{\Req}[5]{\msg{Req}{#1}{#2}{#3}{#4}{#5}}
%%\newcommand{\Resp}[5]{\msg{Resp}{#1}{#2}{#3}{#4}{#5}}
%%\newcommand{\Data}[4]{\msg{Data}{#1}{#2}{#3}{#4}}
%%\newcommand{\Mesg}[4]{\msg{Mesg}{#1}{#2}{#3}{#4}}
%%\newcommand{\fourAngle}[5]{\text{$#1\langle#2,#3,#4,#5\rangle$}}
\newcommand{\ReqCore}[4]{\fourAngle{Req}{#1}{#2}{#3}{#4}}

\subsection{Atomic MSI protocol for 2-levels, no replacement evictions}

XXXXXX: This subsection is probably not going to go into the paper, since it's
too obvious. The definitions have been repeated in the later subsections.

We start with an atomic MSI protocol, with two levels of the memory hierarchy --
the L1 caches and the main memory. In this atomic protocol, the memory
controller can read and write the coherence states as well as the data of each
address present in the L1 caches. As usual, coherence state for an address in
the cache denotes the permissions that the cache has for the address: $M = \{R,
W\}, S = \{R\}, I = \{\}$.

Let us assume that the caches have enough capacity so that no line gets evicted
for the sake of replacement. Lines may however get evicted to enforce the
single-writer property, \ie only one L1 cache can modify a cache line at any
time. Figure~\ref{alg:msi-easy} shows the pseudo-code for the behaviors of the
memory controller and each of the caches to implement the MSI protocol, using
the notations given in Figure~\ref{fig:notation}. 

\begin{figure}
\begin{figure}[H]
\begin{boxedminipage}{\linewidth}
\begin{algorithmic}
\State Pick a request \ReqCore{c}{a}{x}{d} from core to cache $c$ \newline
$s.t.$ $state[c][a] < x$;
\ForAll {$c' \neq c,\; s.t.\; toCompatible(x) < state[c'][a]$}
  \If {$state[c'][a] = M$}
    \State $memory[a] \gets data[c'][a]$;
  \EndIf
  \State $state[c'][a] \gets toCompatible(x)$;
\EndFor
\If {$state[c][a] = I$}
  \State $data[c][a] \gets memory[a]$;
\EndIf
\State $state[c][a] \gets x$;
\end{algorithmic}
\end{boxedminipage}
\subcaption{Behavior of memory controller}
\label{alg:mem}
\end{figure}
\begin{figure}[H]
\begin{boxedminipage}{\linewidth}
\begin{algorithmic}
\State Pick a request \ReqCore{c}{a}{x}{d} from core to cache $c$ \newline
$s.t.$ $state[c][a] \not< x$;
\If {$state[c][a] = M$}
  \State $update(c, a, d)$;
\ElsIf {$state[c][a] = S$}
  \State $respond(c, a, d)$;
\EndIf
\State remove \ReqCore{c}{a}{x}{d}
\end{algorithmic}
\end{boxedminipage}
\subcaption{Behavior of cache $c$}
\end{figure}
\caption{Atomic 2-level MSI protocol: memory controller has instantaneous
access to all states and data of all L1 caches and data in the memory}
\label{alg:msi-easy}
\end{figure}

\begin{figure}
\begin{minipage}{\linewidth}
\begin{tabularx}{\linewidth}{|cX|}
\hline
$toCompatible(s)$ & If the state of one cache is $s$ for address $a$, the
highest state of another cache for the same $a$
$\newline
toCompatible(M) = I,\newline
toCompatible(S) = S,\newline
toCompatible(I) = M
$\\
\hline
$x < y$ & Does state $x$ have fewer permissions that state $y$?
$I < M, I < S, S < M$\\
\hline
$\ReqCore{c}{a}{x}{d}$ & The incoming request for cache $c$ from a core\\
\multicolumn{1}{|r}{$a$} & the address of the cache line\\
\multicolumn{1}{|r}{$x$} & the state of the cache line the request is expecting
(\ie the minimum permissions required)\\
\multicolumn{1}{|r}{$d$} & information about the word offset, the byte enables
and the data (only in the case of store)\\
\hline
$update(c, a, d)$ & Action denoting update of a cache line for address $a$ in
cache $c$ with the incoming request's information $d$\\
\hline
$respond(c, a, d)$ & Action denoting cache $c$ responding to a read request for
address $a$ with the incoming request's information $d$\\
\hline
\end{tabularx}
\subcaption{Abstraction functions}
\end{minipage}

\begin{minipage}{\linewidth}
\begin{tabularx}{\linewidth}{|cX|}
\hline
$state[c][a]$ & coherence state (from set $\{M, S, I\}$) of cache $c$ for
address $a$\\
$data[c][a]$ & data line of cache $c$ for address $a$\\
$memory[a]$ & data line of memory for address $a$\\
\hline
\end{tabularx}
\subcaption{Values stored by a cache for an address}
\end{minipage}
\caption{Notations used in Figure~\ref{alg:msi-easy}}
\label{fig:notation}
\end{figure}

XXXXXX: End of obvious subsection

%
%Let us walk through our framework using the example of an MSI protocol.  We will
%start with a simple system for the sake of explanation.  The system has two
%levels of memory hierarchy -- private L1 caches for each core and a shared
%memory. The L1 caches do not have direct access to the memory, and the memory
%does not have a direct access to the actual states in the  L1 caches, or L1's
%data. Instead, the main memory has direct access to a directory associated with
%it which keeps a version of the L1 cache states for each address. The L1 caches
%do not have direct access to this directory. The directory's version of the L1
%cache states may differ from the actual L1 cache states. We use the terms memory
%and directory interchangeably.
%
%\begin{figure}
%\begin{tabularx}{\linewidth}{|cX|}
%\hline
%$state[c][a]$ & coherence state (from set $\{M, S, I\}$) of cache $c$ for address $a$\\
%$data[c][a]$ & data in cache $c$ for address $a$\\
%\hline \end{tabularx}
%\caption{Values stored by an L1 cache $c$ for an address $a$}
%\label{table:lineinfo}
%\end{figure}
%
%\begin{wrapfigure}{r}{.4\linewidth}
%\centering
%\begin{tabular}{|c|ccc|}
%\hline
%& $M$ & $S$ & $I$\\
%\hline
%$M$ & $=$ & $>$ & $>$\\
%$S$ & $<$ & $=$ & $>$\\
%$I$ & $<$ & $<$ & $=$\\
%\hline
%\end{tabular}
%\caption{$<$ and other relations for MSI states}
%\label{msi<}
%\end{wrapfigure}
%
%Each L1 cache stores the information given in Figure \ref{table:lineinfo}. The
%coherence state for an address denotes the permission that a cache has for the
%address: $M = \{R, W\}, S = \{R\}, I = \{\}$. We can define a $<$ relation for
%coherence state values based on the permissions the states represent. This is
%shown in Figure \ref{msi<}. State $I$ is akin to the cache not having the data
%corresponding to $a$, since it can neither read or write the $a$'s data. Figure
%\ref{table:dirinfo} shows the information stored in the memory and directory. 
%
%\begin{figure}
%\begin{tabularx}{\linewidth}{|cX|}
%\hline
%$dir[c][a]$ & directory's version of the coherence state of cache $c$ for address $a$\\
%$memory[a]$ & data in the memory address $a$\\
%\hline
%\end{tabularx}
%\caption{Values stored by a directory and memory for an address}
%\label{table:dirinfo}
%\end{figure}
%
%The L1 caches and the directory communicate with each using via messages. Each
%L1 cache communicates with its core via 2 buffers -- incoming for requests and
%outgoing for responses. Each L1 cache communicates with the memory directly via
%4 buffers -- two incoming buffers from the memory, one each for requests and
%responses from the memory, and two outgoing buffers into the memory, one each
%for requests and responses to the memory. All these buffers converge into or
%diverge from 4 buffers of the memory -- two incoming buffers, one each for
%requests and responses from the caches and two outgoing buffers, one each for
%requests and responses to the caches (Figure \ref{fig:setup}). All these
%buffers have the FIFO property. Not only that, we will also assume that requests
%sent from a source will never reach its destination till an earlier response
%sent by the same source has been serviced (and hence removed) by the destination.
%%We will show in Section \ref{sec:fifo} about how these network requirements can
%%be realistically implemented.
%
%\subsection{No evictions due to replacement}
%\label{sec:msi-magic}
%
%We will first design an MSI protocol in which the L1 caches have enough
%capacity (or rather associativity) so that no line gets evicted due to
%replacement. Lines may however get evicted to enforce the following invariant
%which is needed for maintaining cache coherence.
%
%%\begin{inv}
%\textbf{Single-writer Invariant}: When a cache has permission to modify data for
%an address ($M$ state), no other cache has read or write permission for that
%address. In other words, the other caches must be in $I$ state.
%\label{singleWriter}
%%\end{inv}
%
%\begin{wrapfigure}{r}{.46\linewidth}
%\begin{subfigure}{.27\linewidth}
%\centering
%\begin{tabular}{|ccc|}
%\hline
%$M$ & $\rightarrow$ & $I$\\
%$S$ & $\rightarrow$ & $S$\\
%$I$ & $\rightarrow$ & $I$\\
%\hline
%\end{tabular}
%\caption*{$toCompatible$}
%\label{toCompat}
%\end{subfigure}
%~~~~~~~~~~~~~~
%\begin{subfigure}{.21\linewidth}
%\centering
%\begin{tabular}{|c|c|}
%\hline
%$M$ & $\checkmark$\\
%$S$ & $\times$\\
%$I$ & $\times$\\
%\hline
%\end{tabular}
%\caption*{$isModified$}
%\label{isModified}
%\end{subfigure}
%\caption{$toCompatible$ and $isModified$ mappings}
%\label{fig:msimap}
%\end{wrapfigure}
%
%Because of the Single-writer invariant, it makes sense to talk about
%compatibility of the states of two caches for an address. Two cache states $x$
%and $Y$ are \emph{compatible} if two caches having the same address in states
%$x$ and $x$ do not violate the Single-writer invariant. We define a mapping
%$toCompatible(x)$ which gives the highest state which is compatible with a given
%state $x$ (Figure \ref{fig:msimap}).
%
%In addition to Single-writer Invariant, maintaining cache coherence also
%requires the following invariant:
%
%%\begin{inv}
%\textbf{Read-from-last-writer}: When a cache reads data for an address, it reads
%the last modified version of the data for the address. If the data for an
%address has not been modified, the cache reads the data from memory.
%%\end{inv}
%
%Function $isModified(x)$ determines if data for an address state $x$ can
%potentially be modified or not (Figure \ref{fig:msimap}).
%
%If a cache $c$ receives a request for an address $a$ from the processor, it can
%service the request straightaway if it has enough permissions for address $a$.
%Otherwise, it has to get to a state for $a$, say $x$, which has enough
%permissions. In other words, it has to \emph{upgrade} its state to $x$. In order
%to maintain the Single-writer Invariant, cache $c$ has to
%ensure that the states of other caches are \emph{compatible} with $x$. But the
%caveat is that $c$ can access only its own state and data, as opposed to that of
%other caches (or the directory).
%
%Cache $c$ has to instead communicate with the directory to ensure that the
%states of other caches are compatible. $c$ must send a message
%\Req{c}{Dir}{a}{x} requesting the directory to tell if it can upgrade its state
%for address $a$ to $x$. We call this an \emph{upgrade request}. Let us now take
%for granted that such a message can be sent (Invariant \ref{csendreq} in Figure
%\ref{DirInv}).
%
%We will also assume that that a cache can upgrade its state to $x$ only after
%receiving and servicing a message from the directory \Resp{Dir}{c}{a}{x}
%permitting it to upgrade its state for address $a$ to $x$ (Invariant
%\ref{cwaitresp} in Figure \ref{DirInv}). We call such a message as an
%\emph{upgrade response}.
%
%We introduce two other kinds of messages: \emph{downgrade request}
%\Req{Dir}{c'}{a}{x} sent from the directory to a cache $c'$ asking $c'$ to
%downgrade its state for address $a$ to $x$, and \emph{downgrade response}
%\Resp{c'}{Dir}{a}{x} sent from $c'$ to the directory informing the directory
%that $c'$ has changed its state $state[c'][a]$ from $x$ to $x$, where $x > x$.
%
%How can the behavior of the directory be designed (in terms of sending and
%receiving messages) to ensure that cache $c$ will get an upgrade response
%eventually. Let's say the invariants given of Figure \ref{DirInv} are
%somehow guaranteed (we have already assumed that invariants \ref{csendreq} and
%\ref{cwaitresp} are guaranteed). With these invariants, let us design the
%behavior of the directory.
%
%\floatstyle{boxed}
%\restylefloat{figure}
%\begin{figure}
%\begin{inv}
%A cache will eventually be able to send an upgrade request.
%\label{csendreq}
%\end{inv}
%\begin{inv}
%Cache $c$ can upgrade $state[c][a]$ from $x$ to $x$ where $x > x$ only after it
%receives and services an upgrade response \Resp{Dir}{c}{a}{x} from the directory.
%\label{cwaitresp}
%\end{inv}
%\begin{inv}
%Just before starting to service an upgrade request \Req{c}{Dir}{a}{x} from $c$,
%$dir[c][a] < x$.
%\label{drecvreq}
%\end{inv}
%\begin{inv}
%The directory will eventually be able to send any downgrade request to any cache
%$c$.
%\label{dsendreq}
%\end{inv}
%\begin{inv}
%The directory can change $dir[c][a]$ from $x$ to $x$ where $x < x$, \ie
%downgrade its directory state only after receiving and servicing a downgrade
%response \Resp{c}{Dir}{a}{x} from $c$.
%\label{dwaitresp}
%\end{inv}
%\begin{inv}
%If the directory has sent a downgrade request \Req{Dir}{c}{a}{x} to cache $c$,
%it will eventually get a downgrade response \Resp{Dir}{c}{a}{z} from cache $c$
%such that $z \le x$.
%\label{drecvresp}
%\end{inv}
%\begin{inv}
%When a cache $c$ sends a downgrade response \Resp{c}{Dir}{a}{x} to the
%directory and $state[c][a] = x$ just before sending the downgrade response, then
%$dir[c][a] = x$ from the time the response was sent by $c$ to the time the
%directory services the response.
%\label{cknows}
%\end{inv}
%\begin{inv}
%When a directory sends an upgrade response \Resp{Dir}{c}{a}{x} to cache $c$ and
%$dir[c][a] = x$ just before sending the downgrade response, then $cache[c][a] = x$
%from the time the response was sent by the directory to the time the $c$
%services the response.
%\label{dknows}
%\end{inv}
%%\begin{inv}
%%Just before receiving a downgrade response from $c$, if $isModified(dir[c][a])$
%%is true, then $c$ would have sent the data along with the downgrade response.
%%\label{drecvdata}
%%\end{inv}
%\begin{inv}
%The directory will eventually be able to send any upgrade response to any cache
%$c$.%It can also send the data to cache $c$ along with the upgrade response.
%\label{dsendresp}
%\end{inv}
%\begin{inv}
%Conservative directory:
%\begin{spacing}{.1}
%\begin{equation*}
%\forall c, a,\; dir[c][a] \ge state[c][a]
%\end{equation*}
%\end{spacing}
%\label{conservative}
%\end{inv}
%\caption{Directory invariants}
%\label{DirInv}
%\end{figure}
%\floatstyle{plain}
%\restylefloat{figure}
%
%%The directory ensures that the states of any two caches for an address are
%%compatible by the following behavior.
%%When the directory receives an upgrade request
%%\Req{c}{Dir}{a}{x}, Invariant \ref{drecvreq} guarantees that $dir[c][a] < x$ and
%%because of Invariant \ref{conservative}, $cache[c][a] le dir[c][a]$, resulting
%%in $cache[c][a] < x$. Unless the directory sends an upgrade response to $c$
%%permitting it to upgrade to $x$, $c$ will remain in state $cache[c][a]$ forever
%On receiving an upgrade request \Req{c}{Dir}{a}{x}, the directory sends a
%downgrade request \Req{Dir}{c'}{a}{toCompatible(x)} to every cache $c'$
%for which $dir[c'][a] > toCompatible(x)$. Such a request can be sent eventually
%because of Invariant \ref{dsendreq}. The incoming upgrade request is not
%removed and no further upgrades requests are processed till an appropriate
%upgrade response is sent back from the directory to $c$.
%
%Because of Invariant \ref{drecvresp}, the directory will eventually receive
%downgrade responses \Resp{c'}{Dir}{a}{z} from each of the requested caches $c'$
%such that $z \le toCompatible(x)$ (since the directory made a downgrade request
%to go to $toCompatible(x)$). Just before servicing the downgrade response, if
%$isModified(dir[c'][a])$ is true (\ie $dir[c'][a] = M$), the directory be
% waits for data to be sent by $c'$. $c'$ knows that the directory's
%version of its state for address $a$, at the time \Resp{c'}{Dir}{a}{z} is
%serviced by the directory will be the same as the state of $c'$ right before it
%sends the downgrade response (by Invariant \ref{cknows}), and hence would know
%if the directory is expecting data by simply checking for
%$isModified(state[c'][a]$ at the time of sending the downgrade response. So $c'$
%can be implemented to send the data along with the response if required. If the
%directory receives data, the memory for address $a$ is updated with the received
%data.
%
%Once all the downgrade responses have been received and serviced, since no
%upgrade responses have been sent to any cache in the meanwhile, we have $\forall
%c' \neq c, dir[c'][a] \le toCompatible(x)$. Because of Invariant
%\ref{conservative}, it is guaranteed that $\forall c' \neq, cache[c'][a] \le
%toCompatible(x)$. The directory now changes $dir[c][a]$ to $x$ and sends an
%upgrade response \Resp{Dir}{c}{a}{x} to $c$. If $dir[c][a] = I$ right before
%changing it to $x$, then by Invariant \ref{dknows}, $cache[c][a]$ is going to be
%$I$ till cache $c$ services the upgrade response. So, in this case, the data
%from the memory for address $a$ is also sent along with the upgrade response.
%Note that such an upgrade response, can always be sent eventually by the
%directory because of Invariant \ref{dsendresp}. Finally, the upgrade response
%from $c$ has been serviced and hence removed by the directory.
%
%The above behavior for the directory ensures that the states of any two caches
%for every address are compatible, thus obeying Single-writer Invariant. This can
%be proven by noting that whenever an upgrade to $x$ request from $c$ is being
%serviced, every other cache is ensured to be in a state $\le toCompatible(x)$
%before an upgrade response is eventually sent to $c$. Similarly, it can be
%easily shown to obey Read-from-last-writer Invariant.
%
%%The above behavior for the directory ensures that the states of any two caches
%%for every address are compatible, as can seen by induction.  This property is
%%true initially because every cache is in state $I$ for every address. If this
%%property is true before the servicing of a request is started, it will be true by the
%%time the request is serviced and an upgrade response is sent. This is because an
%%upgrade response is sent only when every cache other than the one requesting is
%%at a state $\le toCompatible(x)$, where $x$ is the state the requesting cache
%%wants to go to, and $toCompatible(x)$ is the highest state that a cache can be
%%in to be compatible with another cache in state $x$ for an address.
%
%%Whenever the directory receives a downgrade response from a cache $c'$, just
%%before servicing the response if $isModified(dir[c'][a])$ is true, \ie
%%$dir[c'][a] = M$, $c'$ must have been the only cache with permission to modify
%%address $a$. This is because the directory ensures that the states of any two
%%caches for address $a$ is compatible at all times (as shown above). Thus, to
%%ensure the Read-from-last-writer Invariant, the potentially modified data from
%%$c'$ is written back from into the memory on receiving the downgrade response
%%from $c'$ as seen in our design.
%
%The invariants \ref{drecvreq} to \ref{conservative} in Figure \ref{DirInv}
%essentially makes it straight forward to design the behavior of the directory.
%Without these invariants, it will be hard to reason about the correctness of the
%protocol as shown by the following examples.
%
%For the sake of argument, let us assume Invariant \ref{drecvreq} not be
%guaranteed. Consider a scenario in which the cache $c$ which has sent a request
%\Req{c}{Dir}{a}{M}. Before servicing an upgrade response for this request, $c$
%sends another upgrade request \Req{c}{Dir}{a}{S}. Once the directory has
%finished processing the first request, it changes its directory state
%$dir[c][a]$ to $M$. When it starts servicing the second request, it is not easy
%to decide if the directory has to drop the request or to send an upgrade to $S$
%response.
%
%As another example, let us assume that Invariant \ref{dknows} does not hold.
%Let's say the $dir[c][a] = S$ right before the directory sends an upgrade
%response \Resp{Dir}{c}{a}{M} to cache $c$. The directory will not send the data
%because in its version of the state of $c$ for address $a$, cache $c$ already
%has the data. But if the directory was wrong, and at the time cache $c$ services
%the upgrade response, if $state[c][a]$ is $I$, then cache $c$ will no longer
%have the right data, breaking the Read-from-last-writer Invariant.
%
%The difficulty in message passing protocols comes from the fact that it is
%difficult to guarantee the above invariants, even for a 2-level memory hierarchy
%discussed above. as shown in the following scenarios.
%
%%These scenarios show that if the invariants of Figure \ref{DirInv} are violated,
%%then it either breaks cache coherence (by creating deadlocks) or it makes
%%it hard to reason about the behavior of the directory.
%
%%\begin{scen}
%%\emph{Inability to send messages because of network congestion:}
%%The ability of a cache or the directory to send messages (Invariants
%%\ref{dsendreq}, \ref{dsendresp} and \ref{csendreq}) depends upon the
%%availability of free buffers in the network; if the network remains congested
%%all the time, then it would not be possible to send the messages, and the system
%%would deadlock.
%%\end{scen}
%%
%%\begin{scen}
%%\emph{Multiple upgrade requests from the cache for the same address:}
%%If a cache is allowed to make multiple requests to upgrade the state of an
%%address (say to $M$ and then to $S$) before receiving the upgrade responses,
%%then it might violate Invariant \ref{drecvreq} if the requests were processed by
%%the directory in the order they were received. This makes the decision on
%%whether the directory drops the second request or not hard to reason about, as
%%shown above.
%%\end{scen}
%
%\begin{scen}
%\emph{On sending an upgrade request, the cache stops processing downgrade
%requests it gets an upgrade response:}
%Consider a scenario with two caches $c$ and $c'$, with both the caches in state
%$S$ for address $a$. Both the caches get a store request from the processor and
%send \Req{c}{Dir}{a}{M} and \Req{c'}{Dir}{a}{M}, respectively, to the directory.
%Let's say \Req{c'}{Dir}{a}{M} arrives at the directory first. The directory
%sends a downgrade request to cache $c$, but since $c$ does not process any
%downgrade requests till it gets its upgrade response, it will never send back a
%downgrade response to the directory, violating Invariant \ref{drecvresp},
%resulting in a deadlock. Note that this deadlock would occur even if a cache
%waiting for an upgrade response for an address, stops processing downgrade
%requests for other addresses.
%\label{noprocess}
%\end{scen}
%
%\begin{scen}
%\emph{At the directory, downgrade responses from caches are blocked by upgrade
%requests:}
%Consider a scenario where two caches $c$ and $c'$ are both in state $S$ for
%address $a$. $c'$ gets a store request from the processor and sends an upgrade
%request \Req{c'}{Dir}{a}{M} to the directory. The directory receives that
%request, and sends a downgrade request \Req{c}{Dir}{a}{I} to $c$. $c$ sends a
%downgrade response \Resp{c}{Dir}{a}{I} to the directory on receiving the
%downgrade request. But the directory can not service the downgrade response because
%it is blocked by other requests (including \Req{c'}{Dir}{a}{M}). This violates
%Invariant \ref{drecvresp} and hence creates a deadlock.
%\label{scen:block}
%\end{scen}
%
%Let us consider a slightly more complicated system where caches can evict lines
%for replacement before proceeding further.
%
%\subsection{Evictions due to replacement are allowed}
%If a cache starts servicing a request from the core for an address $a$ and has
%no space to accommodate address $a$, it evicts another address $a'$ to make room
%for $a$.
%
%In the previous scenario, the caches didn't have to evict for replacement; a
%cache had to send downgrade responses to the directory only on servicing a
%corresponding request. But in the new scenario, a cache can send a downgrade
%notification voluntarily to the directory on evicting an address. The directory
%has to process this new notification: on servicing the downgrade to $x$
%notification for an address $a$ from cache $c$, the directory changes
%$dir[c][a]$ to $x$, keeping the directory's version of the cache's state
%up-to-date. Since the directory can not distinguish between the downgrade
%notification sent voluntarily and a downgrade response sent because of a
%downgrade request from the directory, we will use the same message to denote
%both.
%
%%We will restrict the family of directory-based protocols that that
%%we discuss to those in which the caches send such notifications.
%
%We will first discuss an example of the behavior of the L1 cache that would
%violate Invariant \ref{conservative}, if the cache is allowed to voluntarily
%send a downgrade response due to eviction.
%
%\begin{scen}
%\emph{Cache sends voluntary downgrade responses indiscriminately:}
%Consider a system with only two caches $c$ and $c'$ (Figure \ref{fig:cross})
%such that $state[c][a] = dir[c][a] = S$, and $state[c'][a] = dir[c'][a] = I$.
%Let $c$ be a non-blocking cache, so it can process requests from the core even
%if there are pending misses. $c$ receives a store request for address $a$ from
%the core and sends \Req{c}{Dir}{a}{M} to the directory. Before it gets back an
%upgrade response from the directory, it receives another load request for
%address $a'$ from the core. Suppose the line for $a'$ is not present in $c$ and
%$c$ has no space for $a'$, it decides to evict address $a$ to make room for
%$a'$. It voluntarily sends a downgrade response \Resp{c}{Dir}{a}{I} to the
%directory. Meanwhile, the directory receives the upgrade request
%\Req{c}{Dir}{a}{M}, and since the directory state of the only other cache is $I$
%for address $a$, the directory sends back an upgrade response. The directory
%then receives the downgrade response \Resp{c}{Dir}{a}{I} and changes $dir[c][a]$
%to $I$. $c$ finally receives the upgrade response \Resp{Dir}{c}{a}{M} and
%upgrades $state[c][a]$ to $M$. This breaks Invariant \ref{conservative}, since
%$state[c][a] = M$ and $dir[c][a] = I$. If the directory gets an upgrade to $M$
%request for address $a$ from $c'$, it will assume that $c$ is in state $I$ for
%address $a$ and send an upgrade to $M$ response to $c'$. This will violate the
%Single-writer invariant, thereby violating coherence.
%\label{indisc}
%\end{scen}
%
%\subsection{Meta-rules that guarantee the invariants of Figure \ref{DirInv}}
%
%We now show meta-rules for implementation of a message-passing directory-based
%protocol in order to guarantee the invariants of Figure \ref{DirInv}; this has
%been formally proven using the Coq theorem prover. We will present some
%intuition behind the proof by showing how these constraints avoid scenarios
%\ref{noprocess}, \ref{scen:block} and \ref{indisc}, though the formal proof is
%beyond the scope of the paper. These meta-rules are amenable to direct
%implementation because each of the meta-rules access only local states.
%
%%We will present the intuition behind the proof by showing how
%%these constraints avoid scenarios \ref{noprocess}, \ref{scen:block} and
%%\ref{indisc}, though the formal proof is beyond the scope of the paper.
%
%\floatstyle{boxed}
%\restylefloat{figure}
%\begin{figure}
%\begin{cons}
%Requests for an address $a$ should not block responses for address $a$, even
%from a different source.
%\end{cons}
%\caption{Meta-rules for blocking}
%\label{block}
%\end{figure}
%\begin{figure}
%\begin{cons}
%Responses to the directory for the same address $a$ from the same cache $c$
%should be serviced by the directory in the order they were sent.
%\end{cons}
%\begin{cons}
%Requests for an address $a$ from a source $n$ (cache or directory) should not be
%serviced by a destination $m$ (directory or cache) before $m$ has serviced all the
%responses for the same address $a$ from the same source $n$ sent earlier. In
%other words, requests between caches and directory should not overtake responses
%between caches and directory if they are sent from the same source to the same
%destination for the same address.
%\end{cons}
%\caption{Meta-rules for ordering}
%\label{order}
%\end{figure}
%
%Figures \ref{block} and \ref{order} gives the meta-rules about which messages
%should not block each other, and the ordering requirements between messages. The
%setup in Figure \ref{fig:setup} and the stated FIFO requirements guarantees
%these meta-Rules. We will show how these meta-rules can be realistically
%realized in Section \ref{sec:fifo}.
%
%\begin{figure}
%\begin{cons}
%A response for address $a$ can be sent by a cache to the directory or by the
%directory to the cache only if the line for $a$ is present in the sending node.
%\end{cons}
%\begin{cons}
%A response from the directory to cache $c$ for address $a$ can be sent only if a
%request from $c$ for $a$ is currently being serviced by the directory.
%\ref{dreqhandle}
%\end{cons}
%\begin{cons}
%If $dir[c][a] = x$, a directory can send a response \Resp{Dir}{c}{a}{x} only if
%$x > x$ and $dir[c][a]$ changes from $x$ to $x$ on sending the response.
%\end{cons}
%\begin{cons}
%If there is a pending response from the directory for an address $a$, a cache
%$c$ can not voluntarily send a downgrade response for address $a$ to the
%directory. But $c$ can send a downgrade response to the directory if it is
%servicing a downgrade request for address $a$ from the directory.
%\ref{creqhandle}
%\end{cons}
%\begin{cons}
%If $state[c][a] = x$, a cache $c$ can send a response \Resp{c}{Dir}{a}{x} only if
%$x < x$ and $state[c][a]$ changes from $x$ to $x$ on sending the response.
%\end{cons}
%\begin{cons}
%On servicing a response \Resp{Dir}{c}{a}{x}, the directory changes $dir[c][a]$
%to $x$.
%\end{cons}
%\begin{cons}
%On servicing a response \Resp{c}{Dir}{a}{x}, cache $c$ changes $state[c][a]$
%to $x$.
%\end{cons}
%\begin{cons}
%$state[c][a]$ can change only if cache $c$ sends a response to the directory or
%services a response from the directory, for address $a$.
%\end{cons}
%\begin{cons}
%$dir[c][a]$ can change only if the directory sends a response to cache $c$ or
%services a response from cache $c$, for address $a$.
%\end{cons}
%\caption{Sending and servicing responses}
%\label{sendRespPre}
%\end{figure}
%
%Figure \ref{sendRespPre} shows the the meta-rules regarding when responses can
%be sent, and the state changes accompanying the sending and servicing of a
%response. In these meta-rules, ``being serviced'' means having started the
%servicing of a request.  For example, if servicing a request requires more
%requests to be sent, ``being serviced'' corresponds to having started sending
%those requests.
%
%Meta-Rule \ref{creqhandle} prevents Scenario \ref{indisc} from happening. A
%cache $c$ should not send a voluntary downgrade response \Resp{c}{Dir}{a}{I} if
%it is waiting for a pending response from the directory, which was being
%violated in that scenario ($c$ evicted address $a$ to replace it with address
%$a'$ while still waiting for an upgrade response for $a$ from the directory).
%
%Meta-Rule \ref{dreqhandle} states that a directory can not voluntarily send an
%upgrade response to a cache. This again leads to Invariant \ref{conservative}
%being violated. The intuition behind having meta-rules \ref{creqhandle} and
%\ref{dreqhandle} is that they prevent ``crossing'' of two response messages as
%shown in Figure \ref{crossing}. This preserves Invariant \ref{conservative}.
%
%\begin{figure}
%\begin{cons}
%A request \Req{Dir}{c}{a}{x} can be sent only if there is no pending response
%from cache $c$ for address $a$, and $x < dir[c][a]$ just before the directory
%sends the request.
%\label{dirreq}
%\end{cons}
%\begin{cons}
%A request \Req{c}{Dir}{a}{x} can be sent only if there is no pending response
%from the directory for address $a$, and $x > state[c][a]$ just before the
%directory sends the request.
%\label{creq}
%\end{cons}
%\begin{cons}
%Requests from the directory for an address $a$ should be serviced even if there
%is a pending response from the directory for any address (either $a$ or other
%addresses).
%\label{cblock}
%\end{cons}
%\caption{Sending and servicing requests}
%\label{sendReqPre}
%\end{figure}
%
%Figure \ref{sendReqPre} shows the meta-rules regarding when requests can be
%sent. If Meta-Rule \ref{cblock} is obeyed, then Scenario \ref{noprocess} will
%be avoided.
