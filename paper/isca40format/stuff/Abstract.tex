We provide a framework for designing hierarchical directory-based
cache-coherence protocols. We reduce the cache-coherence design problem into two
parts: a) a common substrate shared between most directory-based cache coherence
protocols and b) the part specific to each cache-coherence protocol. The common
substrate includes a set of ``\glob{}s'' which provide the framework over which
protocol-specific parts, the ``\policy{}'' can be designed. \glob{}s can be
thought of as a high-level abstraction layer, which allows \policy{} to focus on
high-level properties, instead of worrying about corner cases. \glob{}s include
guarantees like every request from any node getting a response eventually, \etc.
\policy{} includes the actions the directory takes in order to preserve cache
coherence, \viz to preserve the single-writer-at-a-time invariant, and the
reader-reads-last-updated-value invariant. We have formally proven that a set of
(\policy{} independent) ``\local{}s'' guarantee \glob{}s using the Coq theorem
prover. The \local{}s dictate constraints and actions local to a cache or to the
directory as well as properties of the network.

Our framework is flexible and extensible to design multiple hierarchical cache
coherence protocols like MSI, MESI, MOSI, MOESI, \etc -- these protocols are
just different \policy{}s. Our framework has no extra overhead in terms of
state-bits or number of messages transmitted. In fact, the protocols designed
using our framework does not require acks for some of the scenarios where acks
are needed in commonly used protocols \cite{hammer, moesi}. We show that the
overall performance of the protocols designed using our framework is also better
in because of the lack of acks.

%implementations using our framework has no extra overhead in terms of
%state-bits or number of messages passed to service each request. In fact, we
%show that the protocols designed using our framework does not require acks for
%some of the scenarios in which commonly used protocols \cite{hammer, moesi} use
%acks. Our framework
%
%The part specific to each protocol, the ``\policy{}'', includes what actions a
%directory should take on receiving a request from its caches, whether to ask a
%cache to forward the data to another cache, \etc. We have formally proven that
%a set of easy-to-implement ``\local{}s'' guarantee the ``\glob{}s'' rules like
%every request by a cache or a directory will eventually get a response, We
%present a method to develop implementations of hierarchical, message-passing,
%directory-based, cache-coherence protocols. Our method is based on the cache
%system preserving two invariants: a) the directory's notion of the permissions
%for the caches it serves is ``conservative'', \ie the caches do not have more
%permission than what is assumed by the directory, and b) every request by a
%cache or a directory will eventually get a response. We specify an abstract
%protocol, that is, a set of meta rules for state transitions, which has been
%shown, using the Coq theorem prover, to preserve these invariants. We have also
%shown that this abstract protocol can be implemented using $2N + 1$ virtual
%channels where $N$ is the number of levels and each channel can have as few as
%one buffer. 
%
%In this paper, we use the abstract protocol to develop formally proven
%implementations of a number of well-known protocols like MSI, MESI, MOSI and
%MOESI, both in the presence as well as in the absence of point-to-point FIFO
%networks. All these protocols can be expressed in terms of an extra
%``compatibility of state'' check in the state transitions allowed by the
%abstract protocol. We finally show that the protocols we design using our
%methodology have no extra overhead (in terms of state bits or the number of
%messages passed to service each request) compared to the existing
%implementations of these protocols. 
