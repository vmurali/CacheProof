\section{Description of protocols}
\label{sec:protocols}

This has to expand/rewritten, and must relate to our methodology.

\subsection{MSI}
This is the simple protocol. M implies read and write permission, S implies
read permission and I no permission. Whenever a request for M comes from a
cache to the directory, then the directory must ensure that no other cache is in
S or M state. If other caches are in S or M state, the directory sends a
downgrade request to each of these caches to go to I state. Once all the
directory's notion of the other caches' states is I, the directory can send an M
response to the original cache. Similarly, whenever a request for S comes from a
cache to a directory, then the directory must ensure that no othe cache is in M
state. If so, it sends a downgrade request to S state and once every cache is in
S state or I state, the directory sends an S response to the original cache.
Data is transferred from the directory to the cache whenever the directory
responds to a cache in I states. Data is transferred from the cache to the
directory whenever the cache responds transitioning from M to S or I state.

In a hierarchical setting, the directory itself serves as a cache for the higher
level directories, with a notion of state associated with the directory. In this
scenario, when a directory gets a request from its caches, it first ensures that
its own state is not lower than what is requested. If it is lower, it sends a
request to the parent  for upgrade (in addition to sending requests to the other
caches for downgrading)

\subsection{MESI}
This is the same as the MSI protocol except that the caches have a dirty bit.
Thus, data transfer from a cache to a directory (or from a lower level directory
to a higher level directory) is orchestrated only when the bit is dirty. In
fact, all protocols can have a MESI variant in which the dirty bit is kept track
of.

\subsection{MESI, M forwarding}
This involves a cache in M state transferring data to another cache requesting M
or S. The directory, on receiving a request from a cache, checks to see if any
other cache is in M state. If so, it sends a request to the other cache to a)
downgrade its state appropriately (S or I) and b) to send a response to the
first cache with data. Finally the first cache sends a response to the directory
saying that a) the other cache has downgraded and b) the original cache has
received the data. If no other cache was in M state, the directory responds to
the first cache's request like in simple MESI.

\subsection{MOESI, only M or O forwarding}
This involves an extra O state by the cache. A cache in O state is the owner of
the line, and it handles forwarding the data if any other cache requests for S
or M. A cache in O state may not have the line coherent with the memory.
Whenever a cache requests the directory for M state, if another cache has the
line in M or O state (and all other caches are in I state), then the directory
requests the other cache to a) downgrade to I state and b) forward the data to
the first cache. Upon receiving the data, the first cache moves to M state and
sends a response to the directory indicating that a) it has upgraded to M state
and b) the other cache has downgraded to I state.  Whenever a cache requests the
directory for S state, if another cache has the line in M or O state, then the
directory requests the other cache to a) downgrade to S state and b) forward the
data to the first cache. Upon receiving the data, the first cache moves to O
states and sends a response to the directory indicating that b) it has upgrade
to O state and b) the other cache has downgraded to S state. If no other cache
was in M or O state, the directory responds to the first cache's request like in
simple MESI.

\subsection{MOESI, M, O or S forwarding}
This is a variation of the MOESI protocol with the following extra transition.
If a directory receives an M request from a cache, and if one cache is in O or M
state and other caches are in S state, then the directory sends an a request to
every cache to a) downgrade to I, and b) forward the I response to the first
cache.  Simultaneously, the directory sends the count of the number of caches
sending the forward response to the first cache. Once all the forwarded
responses arrive at the first cache, it responds to the directory saying a) all
other caches have downgraded to I and b) it has upgraded to M.
