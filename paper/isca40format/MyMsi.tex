\section{MSI protocol}
\label{sec:GlobalMsi}

Let us walk through our framework using the example of a hierarchical MSI
protocol.

\floatstyle{plain}
\restylefloat{figure}

\begin{figure}
\begin{tabularx}{\linewidth}{|cX|}
\hline
$c.state[a]$ & coherence state (from set $\{M, S, I\}$) of cache $c$ for address $a$\\
$c.data[a]$ & data in cache $c$ for address $a$\\
%$p.dir[n][a]$ & $p$'s notion of the coherence state of $d$ for address $a$. This is meaningful only if $p$ is not an L1 cache\\
$memory[a]$ & data in memory for address $a$\\
\hline
\end{tabularx}
\caption{Notations for information associated with an address in a cache}
\label{notation}
\end{figure}

\begin{figure}
\begin{subfigure}{.43\linewidth}
\centering
\begin{tabular}{|c|ccc|}
\hline
& $M$ & $S$ & $I$\\
\hline
$M$ & $=$ & $>$ & $>$\\
$S$ & $<$ & $=$ & $>$\\
$I$ & $<$ & $<$ & $=$\\
\hline
\end{tabular}
\caption{$<$ and other relations for states}
\label{msi<}
\end{subfigure}
\begin{subfigure}{.31\linewidth}
\centering
\begin{tabular}{|ccc|}
\hline
$M$ & $\rightarrow$ & $I$\\
$S$ & $\rightarrow$ & $S$\\
$I$ & $\rightarrow$ & $I$\\
\hline
\end{tabular}
\caption{$toCompatible$}
\label{toCompat}
\end{subfigure}
\begin{subfigure}{.22\linewidth}
\centering
\begin{tabular}{|c|c|}
\hline
$M$ & $\checkmark$\\
$S$ & $\times$\\
$I$ & $\times$\\
\hline
\end{tabular}
\caption{$isModified$}
\label{isModified}
\end{subfigure}
\caption{MSI state abstraction functions}
\label{funcs}
\end{figure}

The caches are logically organized in the form of a tree
(Figure~\ref{hierarchy}). The leaves of this tree represent the L1 caches,
with each L1 cache connected to exactly one core. The root of the tree is the
Last-Level-Cache, and its connected to the memory. The cache hierarchy is
inclusive, \ie if an address is present in a cache $c$, then it's also present it its
parent $c.parent$. Note that by address we mean cache-line address, not a byte
address or a word address.

Each cache stores the coherence state and the data for each address that it
contains.
%A cache which is not L1 (\ie an internal node in the tree of caches)
%also stores a directory containing its notion of the coherence states of its
%children.
Figure \ref{notation} gives the notation used to represent all this
information. For an L1 cache, the coherence state for an address denotes the
(usual) permission that a core connected to the L1 cache has for the address: $M
= \{R, W\}, S = \{R\}, I = \{\}$. We can define a $<$ relation for coherence
state values based on the permissions the states represent. This is shown in
Figure \ref{msi<}.
%When an address $a$ not present in the L1 cache, then the
%core associated with that L1 can neither read nor write that address, so it's
%equivalent to the state of that address being $I$.

For a non L1 cache, the meaning of coherence states is different. For an
internal cache node $n$, if $n.state[a] = I$, then, the entire sub-tree rooted at $n$
will have coherence state $I$ for address $a$. In particular, no L1 cache in
this subtree will have any permission for $a$. If $n.state[a] = S$, then its
children can either be in $S$ or $I$ state. Finally, if $n.state[a] = M$, then
at most one of its children can be in $M$ state. This guarantees that at most
one of the L1 caches can have write permissions, ensuring the following
invariant:

\begin{theorem}
\textbf{Single-writer Invariant}: When a cache has permission to modify data for
an address, no other cache has read or write permission for that
address.
\label{singleWriter}
\end{theorem}

We define a $toCompatible(x)$ mapping (Figure ~\ref{toCompat}), which specifies
the highest state a cache can be to not violate the Single-writer Invariant, if
another sibling cache is in state $x$ for the same address.

The coherence states also denote the consistency of data in a cache with respect
to its parent. If a cache (both L1 and non-L1) is in state $I$ for an address $a$,
then it does not have the data corresponding to that address. If it is in state
$S$, then its data is consistent with that of its parent. Finally, if it is in
state $M$, then its data is not consistent with that of its parent. If none of
its children are in state $M$, then the data in this cache is the most
up-to-date value. But if one of its children $c$ is in the $M$ state, then the
data in this cache is stale -- the most up-to-date value is present in the
subtree rooted at $c$.

We define a $isModified(x)$ mapping to specify if an address in state $x$ in a
cache can have data different from what's present in its parent. The mappings
$toCompatible$ and $isModified$ abstract away the semantics of the states so
that our framework can be extended to other protocols like MOSI/MOESI.

If a cache having no permission for an address requires read or write
permissions for that address, then the most up-to-date value must be supplied,
\ie:

\begin{theorem}
\textbf{Read-from-last-writer}: When a cache reads data for an address, it reads
the last modified version of the data for the address. If the data for an
address has not been modified, the cache reads the data from memory.
\label{lastRead}
\end{theorem}

Invariants \ref{singleWriter} and \ref{lastRead} together ensure cache coherence
\cite{Patterson}.

We will first describe an atomic version of the hierarchical MSI protocol. Each
of the caches in the hierarchy can be treated like C++/Java objects.
\floatstyle{boxed}
\restylefloat{figure}

\begin{figure}
\small
\begin{algorithmic}
\Proc {\textsc{coreReq}}{$core, a, op, data$}
  \If {$op = Ld$}
    \State \textbf{if} ($this.state[a] < S$)
    \State \;\;\;\; \call{} $this.$\uReq($core, a, S$);
    \State \send{} $this.data[a]$;
  \ElsIf {$op = St$}
    \State \textbf{if} ($this.state[a] < M$)
    \State \;\;\;\; \call{} $this.$\uReq($core, a, M$);
    \State $update(this.data[a], data)$;
  \EndIf
\EndProc
\end{algorithmic}
\caption{L1 cache handling a load or a store request sent by the core by
calling the request handler}
\label{coreHandle}
\end{figure}

In the atomic protocol, a core invokes the L1 cache's request handler
\textsc{coreReq} (Figure \ref{coreHandle}). The L1 cache checks if the request
can already be serviced, if not, it invokes its upgrade request handler \uReq{},
which upgrades the state of the requested address. Finally, in the case of a
load request, the data is sent back to the core, and in the case of a store, the
cache line is updated. Note that the core supplies words or long-words while the
cache line sizes are typically 64 bytes; only the appropriate word in the cache
line is updated in case of a store.

\begin{figure}
\small

\begin{subfigure}{\linewidth}
\begin{boxedminipage}{\linewidth}
\begin{algorithmic}
\Proc {\uReq}{$c , a, x$}
  \If {$c.state[a] < x$}
    \If {$a$ is not present in $this$}
      \If {$this$ has no space for $a$}
        \State Choose address $a'$ for eviction;
        \State \textbf{forall} ($c' \in this.children$)
        \State \;\;\;\; \textbf{if} ($c'.state[a] > I$)
        \State \;\;\;\;\;\;\;\; \call{} $this.$\dReqL($c', a', I$);
        \State \textbf{if} ($this.state[a'] > I$)
        \State \;\;\;\; \call{} $this.$\dResp{}($this.parent, a', I$);
      \EndIf
      \State Allocate a line for $a$;
    \EndIf
    \State // $a$ is already present in $this$
    \State \textbf{forall} ($c' \in this.children$)
    \State \;\;\;\; \textbf{if} ($c'.state[a] > toCompatible(x)$)
    \State \;\;\;\;\;\;\;\; \call{} $this.$\dReqL($c', a', toCompatible(x)$);
    \State \textbf{if} ($this$ is Last-Level-Cache \&\& $this.state[a] = I$)
    \State \;\;\;\; \call{} $this$.\sendMem{}($a$);
    \State \textbf{else}
    \State \;\;\;\; \call{} $this.$\uReqL{}($this.parent, a, x$);
  \State \textbf{if} ($c \neq core$)
  \State \;\;\;\; \call{} $this$.\uResp{}($c, a, x$);
  \EndIf
\EndProc
\end{algorithmic}
\end{boxedminipage}
\caption{$this$ cache's upgrade request handler -- request from one of $this$' children, $c$, to upgrade $c$'s state to $x$ for address $a$ }
\label{atomicUReq1}
\end{subfigure}

\begin{subfigure}{\linewidth}
\begin{boxedminipage}{\linewidth}
\begin{algorithmic}
\Proc {\dReq}{$p , a, x$}
  \If {$this.state[a] > x$}
    \State \textbf{forall} ($c' \in this.children$)
    \State \;\;\;\; \textbf{if} (c'.state[a] > x)
    \State \;\;\;\;\;\;\;\; \call{} $this.$\dReqL($c', a, x$);
    \State \call{} $this.$\dResp($p, a, x$);
  \EndIf
\EndProc
\end{algorithmic}
\end{boxedminipage}
\caption{$this$ cache's downgrade request handler -- request from $this$'
parent, $p$, to upgrade $p$'s state to $x$ for address $a$ }
\label{atomicDReq1}
\end{subfigure}

\caption{MSI protocol}
\label{msi-template}
\end{figure}

Figure \ref{msi-template} gives the methods of a cache to handle the upgrade
requests from its children (or core) and downgrade requests from its parent.
When the upgrade request handler \uReq{} for a cache $this$ is invoked by a
cache $c$ to go to a particular state $x$, $this$ first makes sure that its
other children are compatible with $c$ being in state $x$ for the requested
address. This involves downgrading their states to $toCompatible(x)$. Cache
$this$ has to upgrade its own state before upgrading the state of its child $c$,
if $this$' state is less than the requested state. In this case, $this$ invokes
its parent's handler for upgrade requests. Finally, when state of $this$ is not
less than t$x$ and all its children are compatible with $x$, $this$ can upgrade
$c$'s state, and send the data if $c$'s state was $I$ previously. This is shown
in Figure \ref{atomicUReq1}.

A cache, $this$', downgrade handler \dReq{} can also be invoked (for instance
when its parent wants $this$ to downgrade its state to $x$ because the parent's
upgrade request handler was invoked by some other cache. In this case, $this$
makes sure that each of its child's state is not higher than $x$, by invoking
their respective downgrade methods. When all the children have downgraded, the
cache downgrades its own state to $x$, and sends data back to the parent, if it
has been modified (\ie $isModified(x)$ is $True$).

\begin{figure}
\small

\begin{subfigure}{\linewidth}
\begin{boxedminipage}{\linewidth}
\begin{algorithmic}
\Proc {\uReqL}{$p, a, x$}
  \State \call{} $p.$\uReq($this, a, x$);
\EndProc
\end{algorithmic}
\end{boxedminipage}
\caption{Upgrading address $a$ of $this$ to $x$, by invoking the upgrade handler
of $this$'s parent, $p$}
\label{uReq}
\end{subfigure}

\begin{subfigure}{\linewidth}
\begin{boxedminipage}{\linewidth}
\begin{algorithmic}
\Proc {\dReqL}{$c , a, x$}
  \State \call{} $c.$\dReq($this, a, x$);
\EndProc
\end{algorithmic}
\end{boxedminipage}
\caption{Downgrading address $a$ of $this$'s child $c$ to $x$, by invoking
$c$'s downgrade handler}
\label{dReq}
\end{subfigure}

\begin{subfigure}{\linewidth}
\begin{boxedminipage}{\linewidth}
\begin{algorithmic}
\Proc {\uResp}{$c, a, x$}
  \If {$c.state[a] = I$}
    \State $c.data[a] \gets this.data[a]$;
  \EndIf
  \State $c.state[a] \gets x$;
\EndProc
\end{algorithmic}
\end{boxedminipage}
\caption{$this$ finally upgrading address $a$ of its child $c$ to $x$, and
updating the data of $c$ if necessary}
\label{uResp}
\end{subfigure}

\begin{subfigure}{\linewidth}
\begin{boxedminipage}{\linewidth}
\begin{algorithmic}
\Proc {\dResp}{$p, a, x$}
  \If {$isModified(c.state[a])$}
    \State $p.data[a] \gets this.data[a]$;
  \EndIf
  \State $this.state[a] \gets x$;
\EndProc
\end{algorithmic}
\end{boxedminipage}
\caption{$this$ finally downgrading its state for address $a$ to $x$ and
updating data of its parent $p$ if necessary}
\label{dResp}
\end{subfigure}

\begin{subfigure}{\linewidth}
\begin{boxedminipage}{\linewidth}
\begin{algorithmic}
\Proc {\sendMem{}}{$a$}
  \State $this.data[a] \gets memory[a]$;
  \State $this.state[a] \gets M$;
\EndProc
\end{algorithmic}
\end{boxedminipage}
\caption{Last-Level-Cache $this$ reading data from memory for address $a$}
\label{uResp}
\end{subfigure}

\caption{Methods used in the MSI protocol dealing with other objects}
\label{atomic}
\end{figure}

Figure \ref{msi-template} gives the pseudo-code involved, including the
actions needed to evict a cache line and replace it with another line, in case
there are no free slots available for an address. For reasons that will become
clear in the distributed setting, each cache can directly read or write its own
state and data for each address. It can also directly read it's child's state,
but not directly write it. It can not read it's parents state directly. It can
not directly read or write other cache's data. In order to perform the required
actions indirectly, the cache can invoke other cache's corresponding
\textbf{method}s packaged together as shown in Figure \ref{atomic}. To be
precise we use $this$ explicity even if the object invokes its own methods.

In the case of an L1 cache, once the cache's parent has upgraded its state,
there is no child to upgrade to. Similarly, in the case of a Last-Level-Cache,
the cache simply reads data from the memory if the line is not present in the
cache, since it has no parent whose upgrade request handler it can invoke. All
lines in the Last-Level-Cache are upgraded automatically to be in $M$ state.

\subsection{Refactoring the handler methods}
Figure \ref{msi-template} shows how the request handler methods \uReq{} and
\dReq{} can be refactored to make it amenable to distributed implementation. The
upgrade and downgrade request handlers are not allowed to invoke other cache's
upgrade or downgrade request handlers, or change other cache's states. Instead,
the same calls are made via 5 more local methods listed in Figure \ref{atomic}.
These methods in turn invoke the other caches upgrade and downgrade request
handlers, or change the states of other caches.

\begin{figure}
\small

\begin{subfigure}{\linewidth}
\begin{boxedminipage}{\linewidth}
\begin{algorithmic}
\Proc {\uReqL}{$p, a, x$}
  \State \send{} \Req{this}{p}{a}{x};
  \While {$this.state[a] < x$}
  \State \receive{} \Resp{p}{this}{a}{z};
  \If {$this.state[a] = I$}
    \State \receive{} \Data{p}{this}{a}{d};
    \State $this.data[a] \gets d$;
  \EndIf
  \State $this.state[a] \gets z$;
  \EndWhile
\EndProc
\end{algorithmic}
\end{boxedminipage}
\caption{Upgrading address $a$ of $this$ to $x$, by sending a request to
$this$'s parent $p$}
\label{uReqLocal}
\end{subfigure}

\begin{subfigure}{\linewidth}
\begin{boxedminipage}{\linewidth}
\begin{algorithmic}
\Proc {\dReqL}{$c, a, x$}
  \State \send{} \Req{this}{c}{a}{x};
  \While {$this.dir[c][a] > x$}
  \State \receive{} \Resp{c}{this}{a}{z};
  \If {$isModified(this.dir[c][a])$}
    \State \receive{} \Data{c}{this}{a}{d};
    \State $this.data[a] \gets d$;
  \EndIf
  \State $this.dir[c][a] \gets z$;
  \EndWhile
\EndProc
\end{algorithmic}
\end{boxedminipage}
\caption{Downgrading address $a$ of $this$'s child $c$ to $x$, by sending a
request to $c$}
\label{dReqLocal}
\end{subfigure}

\begin{subfigure}{\linewidth}
\begin{boxedminipage}{\linewidth}
\begin{algorithmic}
\Proc {\uResp}{$c, a, x$}
  \State \textbf{if} ($this.dir[c][a] = I$)
  \State \;\;\;\; \send{} \Data{this}{c}{a}{this.data[a]};
  \State $this.dir[c][a] \gets x$;
\EndProc
\end{algorithmic}
\end{boxedminipage}
\caption{$this$ finally upgrading address $a$ of its child $c$ to $x$, sending a
response to $c$ notifying of the upgrade and sending data to $c$ if necessary}
\label{uResp}
\end{subfigure}

\begin{subfigure}{\linewidth}
\begin{boxedminipage}{\linewidth}
\begin{algorithmic}
\Proc {\dResp}{$p, a, x$}
  \State \textbf{if} ($isModified(this.state[a])$)
  \State \;\;\;\; \send{} \Data{this}{p}{a}{this.data[a]};
  \State $this.state[a] \gets x$;
\EndProc
\end{algorithmic}
\end{boxedminipage}
\caption{$this$ finally downgrading its state for address $a$ to $x$, sending a
response to $c$ notifying of the downgrade and sending data to $c$ if necessary}
\label{dResp}
\end{subfigure}

\begin{subfigure}{\linewidth}
\begin{boxedminipage}{\linewidth}
\begin{algorithmic}
\Proc {\dRespL}{$c , a, x$}
  \If {$isModified(this.dir[c][a])$}
    \State \receive{} \Data{c}{this}{a}{d};
    \State $this.data[a] \gets d$;
  \EndIf
\EndProc
\end{algorithmic}
\end{boxedminipage}
\caption{$this$ handling unsolicited downgrade to $x$ response from $c$ for
address $a$. This happens if $c$ evicted address $a$}
\label{uResp}
\end{subfigure}

\begin{subfigure}{\linewidth}
\begin{boxedminipage}{\linewidth}
\begin{algorithmic}
\Proc {\sendMem{}}{$a$}
  \State \send{} \Req{this}{Memory}{a}{M};
  \State \receive{} \Data{Memory}{this}{a}{d};
  \State $this.data[a] \gets d$;
  \State $this.state[a] \gets M$;
\EndProc
\end{algorithmic}
\end{boxedminipage}
\caption{Last-Level-Cache $this$ sending and receivging data from memory for
address $a$}
\label{uResp}
\end{subfigure}

\caption{Non atomic implementation of MSI protocol}
\label{realistic}
\end{figure}

