\section{Liveness properties}
\label{liveness}

In Section \ref{safety} we showed how the MSI protocol of Figure \ref{trans}
obeys the Store atomicity theorem. In addition to this criteria, we must
establish that the system makes forward progress.

Forward progress can be either of the following:
\begin{enumerate}
\item Weak forward progress: The system never reaches a state where none of the transitions of Figure
\ref{trans} apply or
\item Strong forward progress: At any time, eventually a new request from some processor will be
processed
\end{enumerate}

In order to prove strong forward progress, we need to
establish some more conditions on the strategy to execute the transitions of
Figure \ref{trans} in each cache -- the caches must ensure local
starvation-freedom in the processing of messages when sufficient resources are
available. We do not discuss this issue in this paper and instead focus on 
weak forward progress.

In order to establish that some transition always executes, we prove the
following important invariant.

\begin{inv}
\textit{waitCond}: When a cache $x$ has sent a request to another cache $y$ and
is waiting for a response from $y$, then either $y$ has not yet received the
request, or it has received the request and sent the appropriate response to
$x$.
\end{inv}

This condition ensures that every request gets a response eventually. Using
this invariant, it is easy to see that there will always be some cache in the
system which can either send a request to another cache (ChildSendReq or
ParentSendReq) or process a request (ParentRecvReq, ChildRecvReq or
ChildDropReq) or accept a response from another cache (ChildRecvResp or
ParentRecvResp). We do not give the formal proof here due to lack of space.
