\section{Conclusions and future work}

In this paper, we provided a formal definition of cache coherence via the store
atomicity property and provided a concrete hierarchy-parametric directory-based
MSI protocol whose correct we formally checked via a Coq-based proof. To our
knowledge this is the first instance of a formal proof of a protocol parametric
over the memory hierarchy.

Our protocol is described in a style amenable to mechanical transliteration to a
Bluespec SystemVerilog hardware description allowing us to provide high-quality
verified cache coherence engine. As the protocol and associated proof are robust
to many microarchitecture-oriented changes (\eg{} arbitrary hierarchy and
arbitrary cache size, associativity, and replacement policy choices in each
node), we should be able to produce a performant formally verified hardware
cache coherence system.

Moving this work towards generating a fully verified MSI implementation requires
a number of small steps. First, the simple transliteration of descriptions
between Bluespec and Coq, must be done in a mechanical way. Second, the
compilation of Bluespec to hardware is unverified; to fully trust the
implementation the Bluespec compiler must be verified. Some initial work on this
has already been done~\cite{TDBLP:conf/cav/BraibantC13}. Finally, we must extend
our notion of correctness to apply to the hardware substrate and not just the
level of atomic transactions. Most notably, this requires the some guarantee of
liveness and fairness in handling of messages in the final design. We have
initial Coq-based proofs that with finite buffers a local fairness in
transaction our protocol will have this property, but more work is needed. 

%% We illustrated a direct realization of our proposed proof framework for a
%% concrete MSI cache coherence protocol implementation codifying the intuition
%% behind the system's correctness as simple invariants. While this took
%% significant effort, once done, we showed that many key hardware
%% modifications,\eg{} replacement policies, could be made with a very small
%% addition to the proof. These bits were small enough that an informed hardware
%% designer implementing our protocols can reasonably be expected to provide such
%% additional proofs during refinement, producing a provably correct realistic
%% cache coherence engine implementation.

%% The protocol framework we showed was very flexible, and could easily be
%% extended for variants that go beyond micro-architectural variants, \eg{}
%% extension to MOSI with a relatively small amount of additional reasoning. This
%% gives us good confidence that we can effectively produce frameworks for other
%% classes of protocols, \eg{} update-based protocols, non-inclusive cache
%% hierarchies. Further, nothing about our methodology requires we limit ourselves
%% to cache coherence systems; this approach should naturally extend to any
%% hardware-based system.

%% While fundamentally this work meets our goal of verification of synthesizable
%% hardware specifications, there is still some important mechanization to make
%% this ready for industrial use. Foremost, the simple transliteration between the
%% inductively defined transition system in Coq and the Bluespec hardware
%% description should be fully mechanized.
%% %Further, the proposed extension to provide fairness of transaction scheduling
%% %in the Bluespec compiler should be implemented.
%% Further, to fully guarantee correctness of the resulting hardware, we should
%% also verify the hardware synthesis compiler, possibly using FeSi~\cite{Tomas}.


%%  LocalWords:  associativity atomicity
