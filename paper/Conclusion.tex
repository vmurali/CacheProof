\section{Conclusions and future work}






%% We illustrated a direct realization of our proposed proof framework for a
%% concrete MSI cache coherence protocol implementation codifying the intuition
%% behind the system's correctness as simple invariants. While this took
%% significant effort, once done, we showed that many key hardware
%% modifications,\eg{} replacement policies, could be made with a very small
%% addition to the proof. These bits were small enough that an informed hardware
%% designer implementing our protocols can reasonably be expected to provide such
%% additional proofs during refinement, producing a provably correct realistic
%% cache coherence engine implementation.

%% The protocol framework we showed was very flexible, and could easily be
%% extended for variants that go beyond micro-architectural variants, \eg{}
%% extension to MOSI with a relatively small amount of additional reasoning. This
%% gives us good confidence that we can effectively produce frameworks for other
%% classes of protocols, \eg{} update-based protocols, non-inclusive cache
%% hierarchies. Further, nothing about our methodology requires we limit ourselves
%% to cache coherence systems; this approach should naturally extend to any
%% hardware-based system.

%% While fundamentally this work meets our goal of verification of synthesizable
%% hardware specifications, there is still some important mechanization to make
%% this ready for industrial use. Foremost, the simple transliteration between the
%% inductively defined transition system in Coq and the Bluespec hardware
%% description should be fully mechanized.
%% %Further, the proposed extension to provide fairness of transaction scheduling
%% %in the Bluespec compiler should be implemented.
%% Further, to fully guarantee correctness of the resulting hardware, we should
%% also verify the hardware synthesis compiler, possibly using FeSi~\cite{Tomas}.

