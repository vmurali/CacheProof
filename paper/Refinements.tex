\section{Refinements}
\label{sec:Refinements}

In this section we discuss important design changes that would of interest to
implementors. Most of these require no change at all to our framework only a
change in how certain invariants properties hold/are proven.
%We have formally proven the
%correctness of each of these choices in Coq, but due to space concerns we will
%not discuss them in their full detail. 

\subsection{Implementing unbounded pair-wise communication channels in real hardware}

In order to implement the unbounded pair-wise communication channels that we
assumed in Figure \ref{trans} and throughout Section \ref{safety}, we need
to prove that a) finite buffers are sufficient, b) provide a mapping from
pair-wise communication channels to an interconnect network such as a ring or a
mesh and c) prove that such a mapping does not cause deadlocks.

In order to prove that finite buffers are sufficient to avoid deadlocks, it is
sufficient to observe that the protocol obeys weak forward progress. Therefore
some message will be consumed by the system eventually if it exists.

We will now describe the mapping from pair-wise communication channels to
interconnnect network in greater detail, and informal arguments about why the
mapping is necessary and sufficient.

\begin{enumerate}
\item Each level in the cache hierarchy has a single separate virtual network
for requests from child to parent. This need not maintain FIFO order.
\item Either
\begin{enumerate}
\item there are 2 separate virtual networks overall (shared by all the levels)
for responses from child to parent. The first network carries responses sent by
children who were in state \Mo{} when sending the response and the second
carries response sent by children in state \Sh{} when sending the response.
Neither of these two virtual networks need to maintain FIFO order or
\item there is a single virtual network overall (shared by all the levels) for
responses from child to parent, which maintains a FIFO order.
\end{enumerate}
\item Each level in the cache hierarchy has a single separate virtual network
for both request and response messages from the parent to the child. This
network has to maintain FIFO order between messages with the same
source-destination pair.
\end{enumerate}

The intuition behind this virtual network allotment is as follows. Requests and
responses sent by a parent to its child need to maintain FIFO order, in order
to satisfy Invariant \label{noCross}. Any response message sent to the parent
will be drained by execution of either ParentRecvResp transitions, hence these
messages can share a virtual network across the whole hierarchy. In the case of
responses, FIFO property can be enforced at the protocol itself (in fact the
transitions in Figure \ref{trans} ensures the FIFO property of responses from
child to parent without requiring the network to provide them). The need for 2
separate virtual networks is to avoid head of line blocking in case the
messages go out of order. If FIFO property is maintained, then the messages
will not go out of order, and hence a single virtual network is sufficient.
Since any request can generate new requests (unlike responses), the new
requests should be sent on different virtual networks to prevent deadlocks,
hence the need for separate virtual channels for requests at each level of the
hierarchy.

We do not provide the formal proofs for the above assertions due to lack of space.

\subsection{Protocol}
We will describe two important protocol variations in which all the invariants
that we mention in Section \ref{safety} hold.
