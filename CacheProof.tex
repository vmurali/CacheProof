\documentclass{article}

\usepackage[pdftex]{graphicx}
\usepackage{amsmath}
\usepackage{amsthm}
\usepackage{amssymb}

\newtheorem{theorem}{Theorem}
\newtheorem{lemma}{Lemma}

\begin{document}

\begin{lemma}
Child-state is at x due to parent receiving x-resp and child is at y due to child receiving y-resp $\Rightarrow$ x = y.
\label{CRecvPRecvNetwork}
\end{lemma}

\begin{proof}
Proof by contradiction, assuming x $\neq$ y.

\begin{figure}
\centering
%\includegraphics{CRecvPRecvNetwork.1}
\end{figure}

Child has sent x-resp before receiving y-resp ($\because$ y-resp is the last message causing child transition).

Parent has sent y-resp before receiving x-resp ($\because$ x-resp is the last message causing child-state transition).

$\Rightarrow$ The corresponding y-req should have been sent by the child before sending x-resp ($\because$ FIFO ordering between child to parent req, resp).

$\Rightarrow$ x-resp is not voluntary.

$\Rightarrow$ x-req is sent by parent before sending y-resp ($\because$ x-resp is sent by child, and hence x-req is received by child before receving y-resp and FIFO order ordering between parent to child messages).

But y-resp will not be sent by parent if there is a pending request.

\begin{figure}
\centering
%\includegraphics{CRecvPRecvNetwork.2}
\end{figure}

$\Rightarrow \Leftarrow$
\end{proof}

\begin{lemma}
Child-state is at x due to parent sending x-resp and child is at y due to child sending y-resp, corresponding x-resp was sent by parent before receiving y-resp and y-resp was sent by child before receiving x-resp $\Rightarrow$ x = y.
\label{InsaneNetwork}
\end{lemma}

\begin{proof}
Proof by contradiction, assuming x $\neq$ y.

\begin{figure}
\centering
%\includegraphics{InsaneNetwork.1}
\end{figure}

x-resp has not reached child ($\because$ y-resp is the last message sent by child causing child transition, and child is still in y).

y-resp has not reached parent ($\because$ x-resp is the last message sent by parent causing child-state transition, and child-state is still in x).

Corresponding x-req was received by parent before receiving y-resp.

$\Rightarrow$ x-req was sent by child before sending y-resp ($\because$ FIFO ordering between child to parent req, resp).

$\Rightarrow$ y-resp is not voluntary.

$\Rightarrow$ Corresponding y-req was sent by parent before sending x-resp ($\because$ FIFO ordering between parent to child messages).

$\Rightarrow$ x-resp was sent by parent before receiving y-resp for y-req sent by parent which is not possible ($\because$ parent does not respond when there is a pending request).

\begin{figure}
\centering
%\includegraphics{InsaneNetwork.2}
\end{figure}

$\Rightarrow \Leftarrow$

\end{proof}

\begin{lemma}
Child-state is at x due to parent sending x-resp and child is at y due to child receiving y-resp $\Rightarrow$ y $\le$ x.
\label{CRecvPSend<=}
\end{lemma}

\begin{proof}
Proof by contradiction, assuming x $<$ y.

\begin{figure}
\centering
%\includegraphics{CRecvPSend<=.1}
\end{figure}

Parent sends x-resp only after y-resp ($\because$ x-resp is the last message causing child-state transition).

Child can send corresponding x-req for the x-resp from parent only after child receives y-resp ($\because$ only one pending request is allowed). But y-resp is the last message causing child transition, so child is still at y.

$\Rightarrow$ x-req can not be sent by child since x $<$ y.

$\Rightarrow \Leftarrow$

\end{proof}

\begin{lemma}
Child-state is at x due to parent receiving x-resp and child is at y due to child sending y-resp $\Rightarrow$ y $\le$ x.
\label{CSendPRecv<=}
\end{lemma}

\begin{proof}
Proof by contradiction, assuming x $<$ y.

\begin{figure}
\centering
%\includegraphics{CSendPRecv<=.1}
\end{figure}

Child sends y-resp only after x-resp ($\because$ y-resp is the last message causing child transition).

Child must be in z $>$ y to send y-resp.

$\Rightarrow$ x $<$ z ($\because$ x $<$ y).

$\Rightarrow$ Child must have sent z-req after sending x-resp and transitioning to x, and eventually got back z-resp.

$\Rightarrow$ Parent must have received z-req and sent z-resp, while child-state transitions to z. But this is not possible as x-resp was the last message causing child state transition.

$\Rightarrow \Leftarrow$

\end{proof}

\begin{lemma}
Child-state is at x due to parent sending x-resp and child is at y due to child sending y-resp $\Rightarrow$ y $\le$ x.
\label{CSendPSend<=}
\end{lemma}

\begin{proof}

Proof by contradiction, assuming x $<$ y.

\begin{enumerate}
\item Parent sends x-resp before receiving y-resp and x-resp reaches child after child sends y-resp - Not possible by Lemma \ref{InsaneNetwork}.
\item Parent sends x-resp before receiving y-resp and x-resp reaches child before child sends y-resp.

\begin{figure}
\centering
%\includegraphics{CSendPSend<=.1}
\end{figure}

Child must be in z $>$ y to send y-resp.

$\Rightarrow$ x $<$ z ($\because$ x $<$ y and y $<$ z)

$\Rightarrow$ After child received x-resp, somehow it must reach higher state z. This is possible only if parent sends z-resp, thereby child-state transitioning to z. But x-resp was the last message causing child-state transition.

$\Rightarrow \Leftarrow$

\item Parent receives y-resp before parent sends x-resp.

  \begin{enumerate}
  \item Child sends x-req corresponding to x-resp after sending y-resp.

  \begin{figure}
  \centering
  %\includegraphics{CSendPSend<=.2}
  \end{figure}

  Child is at y $>$ x as y-resp is the last message causing transition for child. Since x $<$ y, child wont send x-req.

  $\Rightarrow \Leftarrow$

  \item Child sends x-req corresponding to x-resp before sending y-resp.

  \begin{figure}
  \centering
  %\includegraphics{CSendPSend<=.3}
  \end{figure}

  Child is at z $<$ x when x-req is sent and z' $>$ y when y-resp is sent.

  $\Rightarrow$ z $<$ z' ($\because$ x $<$ y and z' $>$ y).

  $\Rightarrow$ z'-req must be sent by child before x-resp is received (for x-req). But this is not possible as only one pending request is allowed.

  $\Rightarrow \Leftarrow$

  \end{enumerate}

\end{enumerate}

\end{proof}

\begin{lemma}
Child-state is at x due to parent sending x-resp and child is at y due to child receiving y-resp and parent just received z-req $\Rightarrow$ y = x.
\label{CRecvPSendReq}
\end{lemma}

\begin{proof}

Proof by contradiction, assuming x $\neq$ y.

\begin{figure}
\centering
%\includegraphics{CRecvPSendReq.1}
\end{figure}

Parent sends x-resp only after y-resp ($\because$ x-resp is the last message causing child-state transition).

Child sends corresponding x-req only after receiving y-resp ($\because$ only one pending request is allowed).

$\Rightarrow$ Child can not send z-req before receiving x-resp ($\because$ only one pending request is allowed and x-resp has not reached child yet).

$\Rightarrow \Leftarrow$

\end{proof}

\begin{lemma}
Child-state is at x due to parent receiving x-resp and child is at y due to child sending y-resp and parent just received z-req $\Rightarrow$ y = x.
\label{CSendPRecvReq}
\end{lemma}

\begin{proof}

Proof by contradiction, assuming x $\neq$ y.

\begin{figure}
\centering
%\includegraphics{CSendPRecvReq.1}
\end{figure}

Child sends y-resp only after sending x-resp ($\because$ y-resp is the last message causing child transition).

Parent receives z-req before receiving y-resp ($\because$ x-resp is the last message causing child-state transition).

$\Rightarrow$ Child sent z-req before sending y-resp ($\because$ FIFO ordering between child to parent req, resp).

$\Rightarrow$ y-resp is not voluntary ($\because$ child has pending z-req).

$\Rightarrow$ Parent does not receive z-req till it receives y-resp for the corresponding y-req that parent sent.

$\Rightarrow \Leftarrow$

\end{proof}

\begin{lemma}
Child-state is at x due to parent sending x-resp and child is at y due to child sending y-resp and parent just received z-req $\Rightarrow$ y = x.
\label{CSendPSendReq}
\end{lemma}

\begin{proof}

Proof by contradiction, assuming x $\neq$ y.

\begin{enumerate}
\item Parent sends x-resp before receiving y-resp and x-resp reaches child after child sends y-resp - Not possible because of Lemma \ref{InsaneNetwork}.
\item Parent sends x-resp before receiving y-resp and x-resp reaches child before child sends y-resp.

\begin{figure}
\centering
%\includegraphics{CSendPSend.1}
\end{figure}

y-resp can not be sent voluntarily by child as z-req is pending.

$\Rightarrow$ There is a pending y-req from parent.

$\Rightarrow$ Parent wont receive z-req ($\because$ only one pending request is allowed)

$\Rightarrow \Leftarrow$

\item Parent receives y-resp before parent sends x-resp.

\begin{figure}
\centering
%\includegraphics{CSendPSend.2}
\end{figure}

$\Rightarrow$ There is a pending x-req from child. This is not possible since there are two pending requests from child as child has not received x-resp at this moment.

$\Rightarrow \Leftarrow$

\end{enumerate}

\end{proof}

\begin{lemma}
Child-state is at x due to parent sending x-resp and child was at y before sending z-resp due to child receiving y-resp and parent just received z-resp $\Rightarrow$ y = x.
\label{CRecvPSendResp}
\end{lemma}

\begin{proof}

Proof by contradiction, assuming x $\neq$ y.

\begin{figure}
\centering
%\includegraphics{CRecvPSendResp.1}
\end{figure}

Parent sends x-resp only after y-resp ($\because$ x-resp is the last message causing child-state transition).

Child sends corresponding x-req only after receiving y-resp ($\because$ only one pending request is allowed).

$\Rightarrow$ z-resp is not voluntary.

$\Rightarrow$ Corresponding z-req must be received by child before receiving x-resp ($\because$ x-resp has not yet been received by child when it sent z-resp).

$\Rightarrow$ x-resp is sent by parent between sending z-req ($\because$ FIFO ordering between parent to child messages) and receiving z-resp, which is not possible.

$\Rightarrow \Leftarrow$

\end{proof}

\begin{lemma}
Child-state is at x due to parent receiving x-resp and child is at y before sending z-resp due to child sending y-resp and parent just received z-resp $\Rightarrow$ y = x.
\label{CSendPRecvResp}
\end{lemma}

\begin{proof}

Parent must receive y-resp before receiving z-resp ($\because$ FIFO ordering between child to parent resps). This means that x-resp is not the last message inducing child-state transition.

$\Rightarrow \Leftarrow$

\end{proof}

\begin{lemma}
Child-state is at x due to parent sending x-resp and child is at y before sending z-resp due to child sending y-resp and parent just received z-resp $\Rightarrow$ y = x.
\label{CSendPSendResp}
\end{lemma}

\begin{proof}

Proof by contradiction, assuming x $\neq$ y.

\begin{enumerate}
\item Parent sends x-resp before receiving y-resp.

z-resp reaches parent.

\begin{figure}
\centering
%\includegraphics{CSendPSendResp.1}
\end{figure}

$\Rightarrow$ y-resp reaches parent ($\because$ FIFO ordering between child to parent resps), after x-resp is sent which is not possible.

$\Rightarrow \Leftarrow$

\item Parent receives y-resp before parent sends x-resp.

\begin{figure}
\centering
%\includegraphics{CSendPSendResp.2}
\end{figure}

There is a corresponding x-req that parent receives when sending x-resp. Parent receives x-req before z-resp is received.

$\Rightarrow$ x-req was sent by child before z-resp was sent ($\because$ FIFO ordering between child to parent req, resp).

$\Rightarrow$ z-resp is not voluntary.

$\Rightarrow$ Corresponding z-req was received by child before x-resp was received ($\because$ x-resp was not yet received).

$\Rightarrow$ x-resp was sent by parent between sending z-req ($\because$ FIFO ordering between parent to child messages) and z-resp. This is not possible.

$\Rightarrow \Leftarrow$

\end{enumerate}

\end{proof}


\begin{theorem}
If a request is made for $X$ to transition to $y$, then $X$ will eventually reach $y$ if
\begin{itemize}
\item In every level there is atleast one buffer from parent to child, one buffer for child to parent req, each exclusive to that level and one overall child to parent resp buffer, i.e., each level has a separate virtual channel or network for each of parent-to-child-mesg, child-to-parent-req and one overall child-to-parent-resp for all levels.
\item A request is not left to starve if there exists a local state transition involving that request.
\end{itemize}
\end{theorem}

\begin{proof}
Assume initially that there are infinite buffers.

A leaf can always process any request from its parent and send appropriate response. It doesn't generate any new requests to wait on.

A root can always process any request from its children and send appropriate response. It doesn't generate any new requests to wait on.

Any response sent to a node will be consumed by the node.

Assume for induction that any node whose distance from the leaf less than n eventually generates appropriate response on a request. If n is 1, hypothesis holds (the node is the leaf itself). Assume simultaneosly that any node whose distance from the root less than m eventually generates appropriate response on a request. If m is 1, hypothesis holds (the node is the root itself).

Consider a node whose distance from leaf is n and distance from root is m.

Consider a request to this node from its parent. If it ever gets ``processed'', it can at most generate requests to its children, which will send back appropriate responses (due to induction hypothesis) leading to a good state for the node, and this node will respond its parent. If there is a pending request from child which is getting ``processed'', it can either generate requests to its children, which will send back appropriate responses (due to induction hypothesis) or it can generate request to its parent. But the node can respond to a request from its parent even when it has its own request to parent. So the request from parent will eventually be fully processed leading to an appropriate response.

Consider a request to this node from the child. If it only generates requests to children, then they will send back appropriate responses (by induction hypothesis). Otherwise, it can generate request to its parent. But this will also generate appropriate response by induction hypothesis.

Thus any request will generate an appropriate response if we had infinite buffers.

Let there be just one buffer for parent-to-child, one buffer for child-to-parent-req for each level and one buffer for child-to-parent-resp. The responses are always consumed, so just one buffer overall is suffficient. At the leaf level, any request to a leaf will be processed. So one buffer is sufficient from parent to leaf at that level. Similarly, any request to the root will be processed. So one buffer is sufficient from child to root at that level. (This is true as long as the arbiter avoids starvation). Thus, by induction, one can prove.
\end{proof}

\begin{theorem}
At a moment, child-state is x and child is y $\Rightarrow$ y $\le$ x.
\label{<=}
\end{theorem}

\begin{proof}
Initially this is true (all states are invalid).

If only child-state has changed, it can only go up from invalid.

If only child has changed to x, then it can change only due to a x-resp from parent ($\because$ invalid $<$ x). This means child-state also changes. So this is not possible.

If both child and child-state have changed, Lemmas \ref{CRecvPRecvNetwork}, \ref{CRecvPSend<=}, \ref{CSendPRecv<=} and \ref{CSendPSend<=}, examines all scenarios.
\end{proof}

\begin{lemma}
When parent receives z-req, child-state is x and child is y $\Rightarrow$ y = x.
\label{Req=}
\end{lemma}

\begin{proof}
Initially this is true (all states are invalid).

If both child and child-state have changed, Lemmas \ref{CRecvPRecvNetwork}, \ref{CRecvPSendReq}, \ref{CSendPRecvReq} and \ref{CSendPSendReq}, examines all scenarios.
\end{proof}

\begin{lemma}
When parent receives z-resp, child-state is x and child is y $\Rightarrow$ y = x.
\label{Resp=}
\end{lemma}

\begin{proof}
Initially this is true (all states are invalid).

If both child and child-state have changed, Lemmas \ref{CRecvPRecvNetwork}, \ref{CRecvPSendResp}, \ref{CSendPRecvResp} and \ref{CSendPSendResp}, examines all scenarios.
\end{proof}

\end{document}
